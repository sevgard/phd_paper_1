\documentclass[12pt]{article}

\usepackage{amssymb,amsmath,amsfonts,eurosym,geometry,ulem,graphicx,caption,color,setspace,sectsty,comment,footmisc,caption,pdflscape,subfigure,array,hyperref,enumitem}
\usepackage[round]{natbib}

\normalem

\onehalfspacing
\newtheorem{theorem}{Theorem}
\newtheorem{lemma}{Lemma}
\newtheorem{corollary}[theorem]{Corollary}
\newtheorem{proposition}{Proposition}
\newenvironment{proof}[1][Proof of]{\noindent\textbf{#1} }{\ \rule{0.5em}{0.5em}}

% \newtheorem{hyp}{Hypothesis}
% \newtheorem{subhyp}{Hypothesis}[hyp]
% \renewcommand{\thesubhyp}{\thehyp\alph{subhyp}}

% \newcommand{\red}[1]{{\color{red} #1}}
% \newcommand{\blue}[1]{{\color{blue} #1}}


% \newcolumntype{L}[1]{>{\raggedright\let\newline\\arraybackslash\hspace{0pt}}m{#1}}
% \newcolumntype{C}[1]{>{\centering\let\newline\\arraybackslash\hspace{0pt}}m{#1}}
% \newcolumntype{R}[1]{>{\raggedleft\let\newline\\arraybackslash\hspace{0pt}}m{#1}}

\geometry{left=1.0in,right=1.0in,top=1.0in,bottom=1.0in}
\graphicspath{images/}

\begin{document}

\begin{titlepage}
\title{Employee Referral Programs: Always good?}%\thanks{abc}
\author{Georgii Aleksandrov}%\thanks{abc}
\date{\today}
\maketitle
\begin{abstract}
\noindent \textit{Employee referrals play a crucial role in a firm's recruitment practices. This paper presents a theoretical model that examines the behavior of workers and firms under employee referrals. The model assumes a positive correlation between the abilities of connected workers and postulates that the main driver of employee referrals in the labor market is the social preferences of employees towards their friends. The model considers both voluntary referrals initiated by workers and the implementation of employee referral programs (ERPs), which offer fixed monetary bonuses for successful referrals. The findings indicate that adopting an ERP may not always be beneficial for the employer. For low-qualification jobs and worker networks with strong ties, the firm is better off relying on voluntary employee referrals from current employees. Adoption of an ERP may be beneficial for the firm in labor markets where the job requires high qualifications and when the social ties between referring and referred workers are weak.}\\
\vspace{0in}\\
\noindent\textbf{Keywords:} employee referrals, employee referral programs, social preferences, firm-worker match\\
\vspace{0in}\\
%\noindent\textbf{JEL Codes:} key1, key2, key3\\

\bigskip
\end{abstract}
\setcounter{page}{0}
\thispagestyle{empty}
\end{titlepage}
\pagebreak \newpage




\doublespacing


\section{Introduction} \label{sec:introduction}

Employee referrals are a crucial practice in modern talent acquisition, providing organizations with high-quality candidates. This practice involves current employees recommending individuals from their networks for job opportunities within the organization, and it has gained significant attention for its potential to enhance recruitment outcomes and employee retention.

Numerous studies have explored labor market referrals\footnote{In most studies, the terms "job referrals" and "employee referrals" are used interchangeably. However, in the current paper, the term "job referrals" refers to the informal method of job search used by job candidates, while "employee referrals" denotes the HR practice used by employers to acquire new job candidates. The term "labor market referrals" encompasses both job and employee referrals occurring in the labor market.}, documenting their widespread adoption by organizations. These studies highlight that a substantial proportion of companies, often exceeding 30\%, rely on referrals from current employees to fill job vacancies \citep{holzer1987hiring, neckerman1991hiring, marsden2001social}.

Empirical research demonstrates the benefits of labor market referrals for both job seekers and organizations. Studies based on data from the National Longitudinal Survey of Youth (NLSY) and the European Community Household Panel show that a significant percentage of individuals, up to 87\%, learn about their current jobs through personal contacts and referrals \citep{holzer1987job, pellizzari2010friends}. Referrals provide job seekers with a competitive edge in the hiring process, as workers using referrals have higher chances of being hired \citep{burks2015value}. There is also empirical evidence of the positive wage effect of referrals on starting wages for referred workers \citep{corcoran1980most, montgomery1991social, simon1992matchmaker, galenianos2013learning, brown2016informal, dustmann2016referral}. However, the wage gap between referred and non-referred workers diminishes over time \citep{simon1992matchmaker, galenianos2013learning, brown2016informal, dustmann2016referral}.

The impact of employee referrals on organizational outcomes has also been examined. Research shows that referred workers have lower turnover rates compared to non-referred workers \citep{simon1992matchmaker, dustmann2016referral, brown2016informal}, making referrals an effective tool for reducing turnover costs. Field experiments exploring employee referral programs (ERPs) reveal their direct and indirect effects on employees' tenure and positive impact on retention rates, significantly reducing recruitment, training, and retention costs \citep{friebel2023employee}.

Cross-cultural studies indicate that labor market referrals are prevalent worldwide and not limited to specific geographic regions \citep{alon1997job, wahba2005density, yakubovich2005weak}. Referrals also have an impact across demographic groups, with gender-related effects \citep{corcoran1980most, morrison1990women, lalanne2016old} and influence on racial and ethnic differences \cite{datcher1983impact, green1999racial, loury2006some}. These findings emphasize the significant role of referrals in shaping employment outcomes for diverse populations.

Taking into account the numerous advantages of labor market referrals, it is worth noting that they are not without drawbacks. Referrals can result in labor market segmentation \citep{kugler2003employee}, income inequality \citep{calvo2004effects}, and issues related to diversity, nepotism, and discrimination \citep{pallais2016referential}.

Despite the extensive research on the effects of referrals on labor market outcomes, there still exists a limited theoretical understanding of their underlying mechanisms and dynamics. Previous studies have primarily focused on either the firm's perspective, studying the performance and turnover of referred workers \citep{beaman2012gets, burks2015value, ekinci2016employee, brown2016informal}, or the worker's perspective, exploring the impact of job referrals on wages and other labor market outcomes across different demographic groups \citep{elliott1999social, dustmann2016referral, heath2018firms}.

Moreover, both theoretical and empirical research related to employee referrals often do not explicitly investigate the decisions of firms to implement employee referral programs. \cite{ekinci2016employee}, in developing a theoretical model of career concerns, assumes that current employees never refer their friends without extrinsic incentives provided by the employer, automatically excluding the question of the labor market conditions under which firms choose to implement ERP. Most of empirical studies do not distinguish between voluntary referrals that current employees make without any additional incentives from the firm side and the formal ERPs with financial bonuses paid for referring employees \citep{burks2015value, brown2016informal}. Some research focus solely on voluntary employee referrals \citep{pallais2016referential} or attempt to find the difference between different types of ERPs by comparing labor market outcomes between ERPs with fixed financial bonuses for referring employees and ERPs where the financial incentive of the referring employee depends on the performance of the referred worker \citep{beaman2012gets}.

One of the pioneering studies investigating the conditions under which ERPs are advantageous for firms is the field experiment conducted by \cite{friebel2023employee}. In their research, the authors analyze the effects of different ERP treatments involving varying bonus levels for referring employees, including a no-bonus treatment, in a chain of grocery stores for cashier positions. Although the primary focus of the study is on the indirect effects of referrals, the authors discover evidence indicating that higher bonuses lead to an increase in the number of referrals, while the average quality of referrals (in terms of firm-worker match) decreases with higher bonuses. Furthermore, as long as the bonus is not excessively large, ERPs are found to be beneficial for the firm. The authors attribute the low number of referrals in their study to the less attractive nature of cashier jobs. Empirical research conducted by \cite{friebel2023employee} thus provides a foundation for analyzing the circumstances under which ERPs prove beneficial for firms.

This study aims to bridge the gap in the current referral literature by creating a theoretical framework to analyze why some firms adopt Employee Referral Programs (ERPs) while others do not. The paper addresses the question of which factors influence firms' decisions to implement employee referral programs. To answer this question, a theoretical model is developed to investigate the effectiveness of ERPs under various labor market conditions. Several key assumptions are made to achieve this objective.

Firstly, the productivity of workers in the firm is assumed to be influenced by two types of worker ability: general ability and firm-specific ability. Firm-specific ability captures the variations across firms, which may arise from factors such as firm-specific human capital \citep{becker1962investment, becker1975investment}, differences in the weights firms place on the various activities involved in the job \citep{lazear2009firm}, or the diverse tasks and responsibilities assigned to workers in different firms \citep{gibbons2004task}. It can also be interpreted in terms of an organization-centric approach \citep{bloom2019drives, dessein2022organizational}, which suggests that similar firms adopt different management practices that influence their performance\footnote{The model incorporates an examination of two variations of this assumption. The baseline model, outlined in Section \ref{sec:model}, assumes that labor market participants don't independently observe both the general ability and firm-specific ability of workers. The expanded model, discussed in Section \ref{sec:extension}, enables labor market participants to independently observe both general ability and firm-specific ability of workers.}.

The second assumption, that allows for the integration of both voluntary referrals and employee referral programs in one model is as follows: when a current employee makes a referral, she does not convey any additional information to her employer beyond her own ability levels and the fact of her connection with the job applicant. This assumption allows us to consider referrals as an information elicitation mechanism about the candidate's general ability and the firm-worker specific match through the network of social contacts.

Finally, the core assumption is that employees possess social preferences towards their friends and acquaintances \citep{bandiera2009social, friebel2023employee}. Coupled with the assumption about a positive correlation between the ability levels (both general and firm-specific) of connected workers, as derived from \cite{montgomery1991social}, this serves as the primary driving force behind employee referrals in the model.

The model contributes to the literature on employee referrals in several ways. Firstly, it proposes that a fraction of current employees is willing to make referrals even without external financial incentives from the employer. The likelihood of such voluntary referrals increases with the strength of social ties between referring employees and referred workers, and decreases with higher job requirements (i.e., higher general ability levels of job candidates). The expanded version of the model presented in Section \ref{sec:extension} further suggests that firms benefit most from voluntary referrals under certain conditions: (i) when job qualifications are high; (ii) when social ties between workers are strong; and (iii) when the firms' hiring processes are not efficient.

Secondly, the model also contributes to the emerging body of research on employee referral programs. It claims that adopting an ERP is not always advantageous for employers. More importantly, it identifies the key factors that affect the effectiveness of ERPs from the firm's perspective and thus are crucial for the firm's decision to implement ERPs. Specifically, the firm's decision to implement an ERP depends on: (i) the job requirements in terms of worker qualifications; (ii) the efficiency of the firm's hiring processes in identifying the general abilities of job candidates; and (iii) the strength of the social ties between referring employees and referred workers.

The model demonstrates that an ERP is beneficial for the firm in labor markets where the job requires high qualifications and when the social ties between referring and referred workers are weak. Under these circumstances, the firm faces an under-referral situation, wherein potentially valuable referrals for the firm are not made due to substantial referral costs borne by referring workers. In such cases, the firm benefits from adopting an ERP and hence providing additional incentives for employees to refer individuals from their social networks.

Lastly, by distinguishing between voluntary referrals and employee referral programs, the model provides explanations for disparities in various empirical studies on employee referrals. For instance, \cite{castilla2005social} utilized employer data on employee referral programs to compare the performance of referred and non-referred workers, discovering a positive impact of referrals on the performance of referred workers. In contrast, \cite{pallais2016referential}, who exclusively focused on voluntary referrals in their field experiment, suggested that the performance of referred workers could be higher, lower, or equal to that of non-referred workers. Additionally, \cite{brown2016informal}, without differentiating between voluntary referrals and ERPs, encountered mixed results in terms of referred workers' performance. Specifically, they observed performance enhancements for referred workers only in the high-tech industry, while finding no productivity increase in call centers and trucking. 

The proposed theoretical framework suggests that firms are inclined to adopt ERPs for positions requiring high qualifications and rely on voluntary referrals for low-qualification roles. This assertion, combined with empirical evidence indicating that high-ability workers often refer high-ability candidates \citep{montgomery1992job}, provides a explanatory basis for the divergent empirical outcomes.

The model's further implications suggest that referred workers receive higher initial wages compared to non-referred individuals, and both groups experience wage growth over time, although it's relatively lower for referred workers. These findings are consistent with research by \cite{corcoran1980most}, \cite{montgomery1991social}, and \cite{dustmann2016referral}. Additionally, the model's outcomes align with empirical evidence from studies like \cite{pallais2016referential}, \cite{lalanne2016old}, and \cite{lalanne2021social}, which demonstrate lower turnover rates among referred workers compared to non-referred workers. Furthermore, the model establishes a positive link between the productivity, wage, and tenure of the referring employee and those of the referred candidate. This observation is partially supported by studies like \cite{kugler2003employee}, \cite{lalanne2016old}, and \cite{levati2020impact}. In terms of ERPs outcomes, the model proposes that the number of referrals rises with the magnitude of an ERP bonus, but the quality of these referrals diminishes. This assertion aligns with the findings in \cite{friebel2023employee}. In sum, the model's findings correspond well with existing empirical research on employee referrals, affirming the model's validity.

The rest of the paper is organized as follows. Section \ref{sec:model} introduces the model, including its setup and an analysis of three different scenarios. The first case outlines the baseline model of internal hiring without referrals. The second case considers voluntary referrals by current employees, while the third case explores the model with both voluntary referrals and an employee referral program that offers a monetary bonus to employees whose referrals are hired. In Section \ref{sec:extension}, an extended version of the model is examined, relaxing assumptions about the observability of workers' general and firm-specific abilities. Section \ref{sec:discussion} provides discussions of the model's main results, offering intuitive explanations and suggesting avenues for future research. Finally, Section \ref{sec:conclusion} provides concluding remarks.

% Based on your discussion and conclusion sections, you can structure your introduction in the following way:
% \begin{itemize}
%     \item Introduction to the topic: Start with a general introduction to the concept of employee referrals and their significance in the labor market. Highlight the potential benefits of employee referral programs (ERPs) for firms, such as improved candidate quality and reduced hiring costs. DONE
%     \item Gap in the literature: Discuss the existing research on labor market referrals and point out the limitations or gaps in the current literature. Emphasize that most previous studies have focused on either voluntary referrals or ERPs, but there is a need to consider both aspects together to gain a comprehensive understanding. DONE
%     \item Objective and research question: Clearly state the objective of your study, which is to develop a model of employee referrals that examines the conditions under which the implementation of an ERP is beneficial for firms. Formulate a research question that encompasses the main focus of your investigation. DONE
%     \item Assumptions and framework: Provide an overview of the key assumptions on which your model is built, as discussed in your conclusion section. Mention the two types of worker ability, the positive correlation among connected workers' ability levels, and the social preferences of current employees towards their social contacts. DONE
%     \item Methodology: Briefly explain the methodology you used to develop the model and analyze the effectiveness of referrals under different labor market conditions. Highlight the importance of incorporating both voluntary referrals and ERPs in the analysis. DONE
%     \item Contribution and significance: Discuss the potential contributions of your research to the field. Highlight how your study fills the gap in the literature by providing insights into the combined effects of voluntary referrals and ERPs on firm outcomes. Emphasize the practical implications of your findings for firms and the potential for further empirical research in this area. DONE
%     \item Outline of the paper: Provide a brief overview of the structure of your paper, mentioning the main sections or chapters and their respective content.
% \end{itemize}

% Remember to keep the introduction concise and engaging, capturing the reader's interest and setting the stage for the rest of your paper.

% \section{Literature Review} \label{sec:literature}

\section{Model} \label{sec:model}
This section presents a model of employee referrals, which allows the firm to fill its vacant positions in several ways. In addition to the formal recruitment method of finding candidates from the labor market, the firm can also utilize referrals from its current employees. The model assumes that the firm's production function depends not only on the general ability of the worker but also on a specific component of the worker's ability. This assumption can be interpreted from different perspectives, such as the classical theory of human capital by \cite{becker1962investment}, its modified version by \cite{lazear2009firm}, and the concept of task-specific human capital studied by \cite{gibbons2004task}. 

The model defines voluntary referrals as referrals that are observed by the employer and all other labor market participants. However, the referring workers do not receive any external incentives to refer their contacts. Referring employees do not possess any additional information about referred candidates' abilities, apart from the fact that there is a connection between the referring employees and the referred candidates. However, referrals can still benefit employers due to the correlation between the abilities of referred and referring workers. 

%I consider two variants of the model: the first assumes that firms in the labor market cannot separately observe the general and specific abilities of workers, but only their realized output. Later sections relax these assumptions to allow all labor market participants to observe both general and specific abilities.
The following subsections include the model setup, an analysis of the model with voluntary referrals, and an analysis of the model in which firms can launch employee referral programs with bonuses paid to referring employees if their referrals are hired.

\subsection{Model setup}
This subsection includes the main assumptions, timing of the model, utilities of employees, and firm profits. The main assumptions of the model are listed below.
\begin{enumerate}[label={A}{\arabic*}.]
	\item Production takes place in firms, and there is free entry into production. Firms in the labor market are assumed to have no market power.
	\item A worker's career lasts for $T = 2$ periods. A worker's labor supply is fixed and inelastic, meaning that the worker cannot adjust the amount of time devoted to work or leisure. The labor supply is fixed at one unit for each worker in every period $t_l = 1,2$\footnote{The model assumes that any worker can be employed by the firm for at most two periods. Thus, the time index throughout the paper indicates the period of a particular worker in the firm.}.
	\item Both workers and firms are risk neutral and have a discount factor $\delta = 0$. Hence, the wages are determined by spot-market contracting, following the assumption made in \cite{gibbons1999theory} that there are no benefits to long-term contracts in a setting where workers and firms are risk-neutral and have a discount rate of zero. Another assumption taken from \cite{gibbons1999theory} is that wages that are paid in advance of production, as opposed to one-period piece-rate contracts. If a worker is indifferent between accepting the offer of a firm and accepting the offer from the labor market, the worker always chooses the former, following the assumption in \cite{ekinci2016employee}.
    \item The output of a worker in a firm depends both on the worker's general ability and their firm-specific ability, i.e. $y_{l} = \theta_l + \mu_l$\footnote{For notational clarity, subscript $f$ denoting the firm is dropped.}. This view is consistent with the classic theory of human capital by \cite{becker1962investment}. It can also be interpreted in terms of a skill-weight approach to specific human capital by \cite{lazear2009firm}, where the firm-specific ability of the worker can be seen as a factor that disturbs the market wage of the worker, even when their ability in a particular firm is observed by all market participants. This happens because the output of the worker in the particular firm does not fully reflect their expected output in other firms due to differences in the weights that the market and the firm assign to particular general skills of the worker \footnote{This assumption can as well be interpreted in terms of the concept of job-specific human capital discussed in \cite{gibbons2004task}. According to this view, the difference between the output of a worker in a particular firm and her expected output in the market may arise from variations in job design across different firms.}. 
    \item When a worker enters the labor market at the beginning of their career, $t_l =1$, none of the market participants, including the worker, have direct knowledge of the true values of the worker's general ability, denoted by $\theta_l$, and their firm-specific ability, denoted by $\mu_l$. However, they share common prior knowledge that $\theta_l$ and $\mu_l$ are independently and normally distributed with $\theta_l \sim \mathcal{N}\left( \bar{\theta}, \sigma^2\right)$ and $\mu_l \sim \mathcal{N}\left( 0, 1\right)$ \footnote{This assumption is similar to \cite{ekinci2016employee} assumption in the model of career-concerns. But in the current model, the specific ability of the worker captures differences across firms on the labor market, rather than unobserved innate worker characteristics.}, where $\bar{\theta} \geq 0$ and $\sigma^2 \in (0, \infty)$. The realization of the worker output, $y_l$, is publicly observed by all market participants at the end of the first period $t_l = 1$\footnote{The worker's output, $y_l$, does not have a stochastic component. Therefore, once the output is revealed in period $t_l = 1$, it remains the same if worker $l$ continues to stay in the firm in period $t_l = 2$.}. %The worker and the firm are able to observe the levels of both the worker's general and specific ability in the particular firm \footnote{Note that in the following section of the paper, I will relax this assumption and allow all market participants to observe both the general and specific abilities of the worker in a particular firm. However, for the purpose of this assumption, I assume that only the worker's output is publicly observed.}.
\end{enumerate}

The following assumptions regarding the referral mechanism are based on claims made in previous studies, including \cite{friebel2023employee}, \cite{ekinci2016employee}, and \cite{lester2021heterogeneous}, among others. However, there are also several unique assumptions that are specific to this paper. One of the core differences between this study and others is that there is no information asymmetry between the referring employee and the firm with respect to the ability of the referred candidate. The current employee does not select which person from their contacts to refer and does not have any superior knowledge of the candidate's ability. Instead, the primary source of information from the referral is the connection between the referring employee and the referred candidate, which influences the firm's and the market's beliefs about the productivity of the referred worker.

\begin{enumerate}[label={A}{\arabic*}., resume]
    \item The firm has three options for hiring job candidates. Firstly, it can always hire any number of candidates from the job market. Secondly, if a current employee $i$ voluntarily refers one of her contacts, the firm can hire the referred candidate $j$. Lastly, the firm can also launch an employee referral program with a pecuniary bonus paid to the current employee if her referral is hired.
    \item Only current employees who remain with the firm in the second period (and whose output level is observed by all market participants) are eligible to refer job candidates from their social contacts. Each employee is allowed to refer only one job applicant to their current employer, and each job applicant can be referred by only one employee. The firm has the potential to hire all referred candidates.
    \item Referring a candidate incurs a cost for the current employee $i$. The cost of referral depends on the average level of the general ability of the candidates competing for the vacant position in the firm: $C(\bar{\theta}) \geq 0$. Here, $C(\cdot)$ is a twice continuously differentiable, non-negative, strictly increasing, and convex function. The assumption that referrals are costly is prevalent in the literature on referrals and can be interpreted in various ways. Firstly, when making a referral, the current employee needs to invest her time in searching for a suitable job candidate. Additionally, there are implicit costs associated with factors such as reputation \citep{saloner1985old}, peer pressure \citep{kugler2003employee, heath2018firms}, or the career concerns of the referring employee \citep{ekinci2016employee}. The likelihood of having a successful career increases with the qualifications required for the job. The significance of reputation within the social network and towards current and potential employers increases with the level of job qualification as well. As a result, referral costs tend to rise with the general ability level of the referring employee.
    \item The connected workers are similar to each other in terms of both their general ability and firm-specific ability, i.e., $Corr(\theta_{i},\theta_{j}) = Corr(\mu_{i},\mu_{j})= \rho \in (0,1)$ if $i$ and $j$ know each other. This assumption is commonly found in theoretical and empirical studies on referrals. For example, \cite{montgomery1991social} assumes assortative matching in personal networks, while \cite{ekinci2016employee} assumes that the output of the referring employee serves as a signal of the referred worker's ability and vice versa. Empirical studies by \cite{beaman2012gets}, \cite{burks2015value}, and \cite{lalanne2021social} provide evidence that high ability workers are more likely to refer high ability candidates. Additionally, recent research by \cite{black2020network} shows that referred candidates are more likely to be a good fit for the job and the company culture. They emphasize that the true advantage of referrals for firms lies in obtaining valuable information about the non-cognitive traits of potential candidates.
    \item Another crucial assumption of the model is that employees care not only about their own well-being but also about the well-being of their contacts. The social preference parameter of the current employee $i$ toward her contact $j$ is denoted as $\psi_{ij} \geq 0$ and assumed to be constant for any pairs of $i$ and $j$. This assumption is based on a recent study by \cite{friebel2023employee}, which posits that workers hold social preferences towards friends whom they might refer. It can also be traced back to similar assumptions in \cite{bandiera2005social}, \cite{bandiera2009social} and \cite{beaman2012gets}.
\end{enumerate}

Assuming A4, a worker's specific ability level at their current employer is not relevant to other firms in the labor market. This leads to different beliefs about the worker's expected productivity between the employing firm and other market participants. Since the firm is a price taker (A1), it pays the wage offered to the worker on the market. Therefore, any difference in beliefs between the firm and other market participants regarding the worker's expected output will constitute the firm's profit.

According to Assumption A1, individual firms do not have any power in the labor market. Therefore, the wage of each worker in every period is predetermined by the labor market's beliefs about their expected output, which, in turn, is determined by their general and firm-specific abilities, as stated in Assumption A4. Furthermore, the firm-specific ability of a worker is individual for each firm in the labor market. As a result, the beliefs of one employer about the levels of firm-specific abilities of its workers do not have any impact on the labor market's evaluations of the expected output of these workers.

In other words, the labor market always evaluates the expected output of every worker (and hence their wage) using additional information only about the worker's general ability. Market beliefs about the worker's firm-specific ability never change. Therefore, the wages of workers are given by\footnote{The derivation of the wages can be found in Appendix A}:
\begin{equation}\label{eq_w_m_1}
    w_{m,1} = \mathbb{E}[\theta_{m}] = \bar{\theta}
\end{equation}
\begin{equation}\label{eq_w_m_2}
    w_{m,2} = \mathbb{E}[\theta_{m}|y_{m}] = \bar{\theta} + \frac{\sigma^2}{1+\sigma^2}(y_m - \bar{\theta})
\end{equation}
\begin{equation}\label{eq_w_j_1_y_i}
    w_{j,1} = \mathbb{E}[\theta_{j}|y_{i}] = \bar{\theta}+\rho\frac{\sigma^2}{1+\sigma^2}(y_i-\bar{\theta})
\end{equation}
\begin{equation}\label{eq_w_j_2}
    w_{j,2} = \mathbb{E}[\theta_j|y_j] = \bar{\theta}+\frac{\sigma^2}{1+\sigma^2}(y_j - \bar{\theta}),
\end{equation}
where $w_{m,1}$ denotes the wage paid to a worker hired from the labor market in period $t_m=1$, $w_{m,2}$ denotes the wage paid to a worker hired from the labor market in period $t_m=2$, $w_{j,1}$ denotes the wage paid to a worker referred by a current employee $i$ with the output level $y_i$ in period $t_j=1$, and $w_{j,2}$ denotes the wage paid to a worker referred by a current employee $i$ in period $t_j=2$. Note that due to assumption A5, the wages of workers are not determined by their specific and general abilities, but rather by their expected output denoted by $y_l$. 

In period $t_i = 2$, the current employee $i$ faces the choice of referring her contact $j$ to her employer. This choice is denoted as $r_i$, where $r_i = 0$ signifies no referral, $r_i = 1$ indicates referring candidate $j$ without being hired, and $r_i = 2$ represents a successful referral resulting in candidate $j$ being hired. The utility function of employee $i$ at time $t_i = 2$ incorporates her wage, $w_{i,2}$; the expected wage of her contact $j$, $w_{j,1}$ if referred ($r_{i} \in \lbrace 1,2 \rbrace$) or $w_{m,1}$ if not referred ($r_{i}=0$); the social preference parameter towards her friend, $\psi_{ij}$; and the costs of referral if $r_{i} \in \lbrace 1, 2\rbrace$. It can be represented as:
\begin{equation}\label{eq:utilitiy}
        U_{i}(y_i) = 
        \begin{cases}
		w_{i,2} + \psi_{ij} w_{j,1} - C(\bar{\theta}) & \text{if } r_i = 2 \\
		w_{i,2} + \psi_{ij} w_{m,1} - C(\bar{\theta}) & \text{if } r_i = 1 \\
        w_{i,2} + \psi_{ij} w_{m,1} & \text{if } r_i = 0
        \end{cases}
\end{equation}

% Note that the expected wage of candidate $j$ in the second period, as perceived by current employee $i$, remains the same regardless of whether candidate $j$ is hired by the firm or not. This is due to the fact that the current employee's belief about the expected value of $\theta_j$ in period $t_j = 2$ is the same in case $j$ is hired by the firm or finds another job on the labor market. In other words, the relationship between the current employee $i$ and candidate $j$ alters the current employee's belief about the general ability of her friend $j$ (and thus his expected wage in period $t_j = 2$) regardless of where he works\footnote{This holds true due to assumption A9, which states the equality between $Corr(\theta_{i},\theta_{j})$, $Corr(\mu_{i},\mu_{j})$, and thus $Corr(y_i,y_j)$. The formal proof is presented in Appendix A.}.

The firm's profit is equal to the difference between the expected output and the wage paid to the worker.
\begin{equation}\label{eq_pi_m_1}
    \pi_{m,1} = \mathbb{E}[y_m]-\mathbb{E}[\theta_m] = 0
\end{equation}
\begin{equation}\label{eq_pi_m_2}
    \pi_{m,2} = y_m - \mathbb{E}[\theta_m|y_m] = \frac{y_m-\bar{\theta}}{1+\sigma^2}
\end{equation}
\begin{equation}\label{eq_pi_j_1}
    \pi_{j,1} = \mathbb{E}[y_j|y_i]-\mathbb{E}[\theta_j|y_i] = \frac{\rho}{1+\sigma^2}(y_i-\bar{\theta})
\end{equation}
\begin{equation}\label{eq_pi_j_2}
    \pi_{j,2} = y_j - \mathbb{E}[\theta_j|y_j] = \frac{y_j-\bar{\theta}}{1+\sigma^2},
\end{equation}
where $\pi_{m,1}$ represents the expected profit generated by a worker $m$ hired from the labor market in period $t_m = 1$, $\pi_{m,2}$ represents the expected profit generated by a worker $m$ hired from the labor market in period $t_m = 2$, $\pi_{j,1}$ represents the expected profit generated by a worker $j$ referred by a current employee in period $t_j = 1$, and $\pi_{j,2}$ represents the expected profit generated by a worker $j$ referred by a current employee in period $t_j = 2$.

The timing of the model is as follows: At the beginning of each period, the firm with a vacant position checks whether any current employee $i$, who will remain in the firm for the second period and has an output level $y_i$ generated in the previous period, is willing to refer one of her social contacts $j$. If there is no referral from current employees, the firm hires a candidate from the labor market and pays them\footnote{Pronouns they/them are used for the non-referred worker $m$, she/her - for the referring employee $i$, and he/him - for the referred worker $j$.} the wage $w_{m,1}$. If there is a referral $j$ made by the current employee $i$, the firm decides whether to hire the referred candidate $j$ or a labor market candidate $m$ and pays the hired worker $l \in \lbrace j,m \rbrace$ the wage $w_{l,1}$. At the end of period $t_l = 1$, all market participants observe  the worker's output $y_{l,1}$. % Moreover, the worker $l$ and the firm observe as well his general and specific ability $\theta_l$ and $\mu_l$. 

At the beginning of period $t_l =2$, the firm decides whether to prolong the contract with the worker $l$ based on their output level $y_l$. The firm prolongs the contract and pays the wage $w_{l,2}$ if its expected profit in the second period exceeds the expected profit from employing a labor market candidate in the first period, i.e. $\mathbb{E}[\pi_{l,2}|y_{l,1}] \geq \pi_{m,1}$. Otherwise, the firm hires a labor market candidate with the wage $w_{m,1}$; so worker $l$ leaves the firm and accepts the labor market offer with the wage $w_{l,2}$. At the end of period $t_l = 2$, worker $l$ retires from the firm.

\subsection{Analysis of the case without referrals}

In the case where there are no referrals, the firm will hire a labor market candidate, denoted as $m$, in the first period and pay them the market wage of $w_{m,1} = \bar{\theta}$. Because the firm has no power on the labor market, its expected profit in the first period is zero: $\pi_{m,1} = 0$.

At the beginning of the second period, the wage of the worker $m$ is adjusted according to their performance in the first period: $w_{m,2} = \mathbb{E}[\theta_m|y_m] = \bar{\theta} + \frac{\sigma^2}{1+\sigma^2}(y_m-\bar{\theta})$. The firm's expected profit in the second period, denoted as $\pi_{m,2}$, is equal to $\frac{y_m-\bar{\theta}}{1+\sigma^2}$. Therefore, the firm will only be interested in continuing the contract with worker $m$ if their output in the first period, denoted as $y_m$, is above the average general ability level $\bar{\theta}$. If $y_m$ is below $\bar{\theta}$, the firm can hire another labor market participant with an expected profit of $\pi_{m,1}=0$.

To summarize, the expected profit of the firm in the second period is given by:
\begin{equation}
    \mathbb{E}[\pi_{m,2}]= 
    \begin{cases}
        0 & \text{ if } y_m < \bar{\theta}\\
        \frac{1}{\sqrt{1 + \sigma^2}}\frac{\phi(0)}{1-\Phi(0)} & \text{ if } y_m \geq \bar{\theta},
    \end{cases}
\end{equation}
where $\phi(\cdot)$ and $\Phi(\cdot)$ represent the probability density function and cumulative distribution function of the standard normal distribution. The equilibrium behavior in the case without referrals is formally stated in Proposition \ref{prop:eq_no_ref}\footnote{Proofs of propositions, lemmas, and corollaries are in Appendix A}:

\begin{proposition}\label{prop:eq_no_ref}
In the absence of referrals by current employees, the firm's hiring and wage decisions, and its profits are determined as follows:
    \begin{enumerate}[label={\roman*})]
        \item At the start of period $t_m = 1$, the firm hires a worker $m$ from the labor market and pays them a wage of $w_{m,1} = \bar{\theta}$. The expected profit of the firm in period $t_m = 1$ from hiring worker $m$ is zero.
        \item At the start of period $t_m = 2$, the firm either retains worker $m$ and pays them a wage of $w_{m,2} = \bar{\theta}+ \frac{\sigma^2}{1+\sigma^2}(y_m-\bar{\theta})$, if their output $y_m$ in period $t_m = 1$ is greater than or equal to $\bar{\theta}$, or lets worker $m$ leave to accept an outside offer of $w_{m,2}$, if $y_m < \bar{\theta}$, while the firm hires another labor market candidate $m'$ and pays them $w_{m,1} = \bar{\theta}$. The expected profit of the firm in period $t_m = 2$ is $\Pi_{m,2} = \frac{\phi(0)}{\sqrt{1+\sigma^2}}$.
	\end{enumerate}
\end{proposition}

The described simplistic model generates several claims that will be used in further analysis. Specifically, the model predicts that the wage of a worker who remains with the firm in the second period is higher than in the first period. Indeed, the wage of the worker increases with their tenure in the firm: $w_{m,1} = \bar{\theta} \leq \bar{\theta}+ \frac{\sigma^2\lambda(0)}{\sqrt{1+\sigma^2}} = \mathbb{E}[w_{m,2}|y_m\geq \bar{\theta}]$, where $\lambda(\cdot) = \frac{\phi(\cdot)}{1-\Phi(\cdot)}$ is the inverse Mills ratio. This prediction is consistent with empirical evidence presented in various studies, such as \cite{medoff1980experience}, \cite{mincer1981labor}, and \cite{topel1991specific}. One can also think of the first period in the firm as a probationary period for the job candidate. If the firm is satisfied with the worker's output, it will prolong their contract; otherwise, it will look for another candidate to fill the vacant position.

Another important result generated by the model is that the firm's overall expected profit from a particular worker decreases with the variance of general ability, denoted by $\sigma^2$. The overall expected profit of the firm from hiring worker $m$ is equal to:
\begin{equation}\label{eq_Pi_m}
    \Pi_m = \pi_{m,1}+P(y_m \geq \bar{\theta})\mathbb{E}\left[\pi_{m,2}|y_m \geq \bar{\theta}\right] = \frac{\phi(0)}{\sqrt{1+\sigma^2}}
\end{equation}
This means that the more precise and efficient the screening mechanism of job candidates available on the labor market, the higher the employer's profit will be.

\subsection{Analysis of the case with voluntary referrals}

In this subsection, the model allows a current employee who stays with the firm for the second period, to refer one of her contacts for a vacant job position in the firm. 

It is important to note that the output level of the current employee who decides to make a referral cannot be below average: $y_i \geq \bar{\theta}$. Otherwise, the firm would not prolong the contract with the employee $i$ for period $t_i = 2$, according to Proposition \ref{prop:eq_no_ref}. This truncated-from-below distribution of the current employee's output, together with a positive correlation between the abilities of the referring and referred workers ($\rho \in (0,1)$), potentially allows the employer to extract additional benefits from hiring referred job candidates (especially in their first period of work). However, the decreased variance of the truncated distribution can negatively affect the expected profit of the firm in the second period of employment of referred workers, in particular when the output of the referring employees is close to the mean. In other words, the firm faces a constraint when deciding whether to hire a candidate $j$ referred by the current employee with an output level $y_i$. The firm's expected profit from employing this referred candidate should be at least as high as the expected profit from hiring a labor market candidate $m$: $\Pi_j (y_i) \geq \Pi_m = \frac{\phi(0)}{\sqrt{1+\sigma^2}}$.

The difference between the firm's expected profit from employing a referred candidate and the expected profit from hiring a labor market candidate is denoted as $\Delta\Pi_{j,m}(y_i)$ and equal to\footnote{The derivation of the profit difference is provided in the proof of Lemma \ref{lemma:y_star_existence} in Appendix A.}: 
\begin{equation}\label{eq:profit_dif}
    \Delta\Pi_{j,m}(y_i) =  \frac{\rho\left(y_i-\bar{\theta}\right)}{1 + \sigma^2}\left(1+\Phi\left(\alpha(y_i)\right)\right)
    + \sqrt{\frac{1-\rho^2}{1+\sigma^2}}\phi\left(\alpha(y_i)\right) - \frac{\phi(0)}{\sqrt{1+\sigma^2}},
\end{equation}
where $\alpha(y_i) = \frac{\rho\left(y_i - \bar{\theta}\right)}{\sqrt{(1-\rho^2)(1+\sigma^2)}}$. The profit difference $\Delta\Pi_{j,m}(y_i)$ is an increasing function of the current employee's output $y_i$. Therefore, the firm will hire the referral $j$ only if the referring employee's output is higher or equal to some threshold $y_i \geq y^*$. Moreover, $\Delta\Pi_{j,m}(\bar{\theta}) < 0$ and $\lim_{\bar{\theta} \rightarrow \infty} \Delta\Pi_{j,m}(\bar{\theta}) = \infty$. Therefore, there exists a unique threshold $y^*$ which is higher than the average expected output of the job candidates on the labor market. This result is formally stated in the following lemma:
\begin{lemma}\label{lemma:y_star_existence}
    Consider the model where a current employee $i$ who stays with the firm in period $t_i = 2$ is able to refer one of her contacts $j$ for a vacant job position in the firm. The firm will hire the referral $j$ only if the current employee's output is greater than or equal to a certain threshold: $y_i \geq y^*$. This threshold is greater than average level of general ability ($y^* > \bar{\theta}$) and is determined by solving the following equation:
    \begin{equation}
        \Delta\Pi_{j,m}(y^*) = 0
    \end{equation}
\end{lemma}

% The profit difference $\Delta\Pi_{j,m}(y_i)$ depends on the output of the current employee $y_i$, and not on her general or specific ability levels $\theta_i$ and $\mu_i$. Moreover, $y^* > \bar{\theta}$, therefore, there exist some employees who stayed in the firm for the second period but the firm is not willing to hire their referrals. Those employees have the following output levels: $ y_i \in \left[\bar{\theta}, y^*\right)$.

All labor market participants, including the referring employee herself, can observe the threshold $y^*$. This means that the current employee knows whether the firm will hire the referred job candidate at the moment of making the decision to refer. The form of the referring employee's utility function (\ref{eq:utilitiy}) establishes that it is worse to refer a candidate that won't be hired by the firm than to not refer the candidate at all. As a result, the current employee never refers a job candidate who won't be hired by the firm, implying that $P(r_i = 1) = 0$. %Moreover, since $y^* > \bar{\theta}$, there are some employees who stayed with the firm for the second period but whose referrals are not considered by the firm. These employees have output levels within the range of $ y_i \in \left[\bar{\theta}, y^*\right)$ and never make referrals.

From the other side, when the current employee makes a decision to refer one of her contacts, she tries to maximize her own utility $U_i(y_i)$. To achieve this objective, she compares her expected utility under successful referral, denoted as $U_i(y_i, r_i = 2)$, with her utility in case she decides not to refer her contact, denoted as $U_i(y_i, r_i = 0)$. Her utility in the case where the referred candidate $j$ is hired should be greater than her utility under no referral: $U_i(y_i, r_i=2) \geq U_i(y_i, r_i = 0)$. Denote the difference in the current employee $i$'s utilities as $\Delta U_i(y_i)$. Then it can be expressed as follows:
\begin{equation}
    \Delta U_i (y_i) = \psi_{ij}\left( w_{j,1}- w_{m,1}\right) - C\left( \bar{\theta} \right) = \frac{\psi_{ij}\rho\sigma^2\left( y_i - \bar{\theta} \right)}{1+\sigma^2} - C\left( \bar{\theta} \right)
\end{equation}
It is easy to observe that if the social preference parameter $\psi_{ij}$ is positive,  $\Delta U_i(y_i)$ is increasing in $y_i$, $\Delta U_i(\bar{\theta}) = -C\left(\bar{\theta}\right)<0$, and $\lim_{y_i \rightarrow \infty}{\Delta U_i(y_i) = \infty}$. Therefore, there exists a unique threshold $\Tilde{y}>\bar{\theta}$ such that $\Delta U_i(\Tilde{y})=0$. This result is formally stated in Lemma \ref{lemma:y_tilde_existence}.
\begin{lemma}\label{lemma:y_tilde_existence}
    Consider the model where a current employee $i$ who stays with the firm in period $t_i = 2$ is able to refer one of her contacts $j$ for a vacant job position in the firm.  The current employee will never make a referral if her social preference parameter $\psi_{ij}$ is equal to zero. However, if $\psi_{ij}$ is positive, the current employee will make a referral only if her output level is greater than or equal to the higher of two thresholds: $y_i \geq \max \lbrace y^*, \Tilde{y} \rbrace$. Here, $y^*$ satisfies $\Delta\Pi_{j,m}(y^*) = 0$, and $\Tilde{y}$ is given by:
    \begin{equation}
        \Tilde{y} = \bar{\theta}+\frac{C(\bar{\theta})(1+\sigma^2)}{\psi_{ij}\rho\sigma^2}
    \end{equation}
\end{lemma}

Lemma \ref{lemma:y_tilde_existence} establishes a crucial result in the model, stating that the social preferences of current employees towards their social contacts, combined with the specific ability component in the firm's production function, drive voluntary referrals in the labor market. In other words, if employees are indifferent to the well-being of their friends, they have no incentive to make referrals. This finding is consistent with the studies of \cite{bandiera2009social} and \cite{friebel2023employee}, which explore situations where incumbent workers exhibit altruistic behavior towards their friends.

Lemma \ref{lemma:y_tilde_existence} also highlights that an altruistic employee (i.e., one with a positive social preference parameter towards her friend, $\psi_{ij} > 0$) faces two distinct thresholds when deciding whether to refer a job candidate $j$. Both of these thresholds are higher than the average worker's output $\bar{\theta}$. Thus, the probability for a current employee $i$ who has stayed with the firm for the second period and whose output is above average ($y_i \geq \bar{\theta}$) to choose not to make a referral is always positive. To put it differently, there will always be some current employees who choose not to make referrals, even if their output is above average. %, i.e., $P \left(y_i \in [\bar{\theta}, \max \lbrace y^*, \Tilde{y} \rbrace) \right) >0$.

Regarding the likelihood of making a referral, the model produces the following implications. It's evident that $\Tilde{y}$ increases with the average level of general ability, $\bar{\theta}$, and decreases with the social preference parameter, $\psi_{ij}$, the variance of the general ability level, $\sigma^2$, and the correlation coefficient $\rho$. %Therefore, the probability that $y_i$ is higher than $\Tilde{y}$ decreases with $\bar{\theta}$ and increases with $\rho$, $\sigma^2$, and $\psi_{ij}$.

$\Delta\Pi_{j,m}(y_i)$ is independent of $\psi_{ij}$ and decreases with $\bar{\theta}$, implying that the value of $y^*$ does not vary with $\psi_{ij}$ and increases with $\bar{\theta}$\footnote{The proof that $\Delta\Pi_{j,m}(y_i)$ increases with $\bar{\theta}$ is analogous to the proof of $\Delta\Pi_{j,m}(\Tilde{y})$ increasing with $\bar{\theta}$, as demonstrated in the Proof of Lemma \ref{lemma:erp_existence}, and therefore is omitted here.}. Since $\Tilde{y}$ decreases with $\psi_{ij}$ and increases with $\bar{\theta}$, it follows that $P\left( y_i \geq \max\lbrace\Tilde{y}, y^* \rbrace \right)$ is non-decreasing in $\psi_{ij}$ and decreasing in $\bar{\theta}$. In simpler terms, the probability of a voluntary referral doesn't decrease with the social preference parameter $\psi_{ij}$ and decreases with the worker's average general ability level. This result can be interpreted as follows: the lower the average qualifications of workers and the stronger the social ties between the current employee and her contact, the higher the probability of a voluntary referral occurring.

Moreover, Lemma \ref{lemma:y_tilde_existence} raises the question of which threshold is binding under what circumstances, i.e., when $y^* \geq \Tilde{y}$ and vice versa. This is a crucial question for this paper, as the answer to it provides insight into the conditions under which the employee referral program (ERP) is beneficial for the firm. Notice, $y^*$ is the firm's threshold that determines whether the referred candidate will be hired. Therefore, if $y^* \geq \Tilde{y}$, there are employees who would like to refer their friends\footnote{These employees' output lies in the interval $[\Tilde{y}, y^*)$.}, but their referrals are not beneficial for the firm. However, if $y^* < \Tilde{y}$, the firm experiences a situation where it would be willing to hire more referred candidates, but the current employees are not willing to refer them. In other words, the firm experiences under-referral from the employees' side. In this situation, the firm could potentially benefit by incentivizing current employees whose referrals increase the firm's profit but are not willing to refer voluntarily.

Further analysis of the employee referral program existence requires to establish the equilibrium behavior in the model with voluntary referrals, which is stated in Proposition \ref{prop:eq_vr}:
\begin{proposition}\label{prop:eq_vr}
    In the model where a current employee $i$ who stays with the firm in period $t_i = 2$ is able to refer one of her contacts $j$ for a vacant job position in the firm, the current employee's referral decision, the firm’s hiring and wage decisions, and its profits are determined as follows:
    \begin{enumerate}[label={\roman*})]
		\item At the start of period $t_j = 1$ the firm hires referred worker $j$ and pays him wage $w_{j,1} = \bar{\theta}+\frac{\rho \sigma^2}{1+\sigma^2} \left(y_i - \bar{\theta}\right)$ if the output level of the referring employee $y_i \geq \max \{ \Tilde{y}, y^* \}$. Otherwise, the firm hires the labor market candidate $m$ and pays them $w_{m,1} = \bar{\theta}$. The expected profit of the firm from hiring referral $j$ by the current employee $i$ with output level $y_i$ in $t_j = 1$ is equal to $\pi_{j,1} = \frac{\rho}{1+\sigma^2}\left(y_i - \bar{\theta}\right)$.
        \item At the start of period $t_j = 2$ the firm retains referred worker $j$ and pays him wage $w_{j,2} = \bar{\theta} + \frac{\sigma^2}{1+\sigma^2}\left( y_j - \bar{\theta} \right)$ if $y_j \geq \bar{\theta}$. Otherwise, the worker $j$ leaves the firm and accepts the outside offer with the wage $w_{j,2}$; the firm hires the labor market candidate and pays them $w_{m,1} = \bar{\theta}$. The expected profit of the firm from the worker $j$ referred by the current employee with the output level $y_i$ in $t_j = 2$ is equal to $\Pi_{j,2} = \frac{\rho\left(y_i-\bar{\theta}\right)}{1 + \sigma^2}\Phi(\alpha(y_i))
        +\sqrt{\frac{1-\rho^2}{1+\sigma^2}}\phi(\alpha(y_i))$, where $\alpha (y_i) = \frac{\rho\left(y_i - \bar{\theta}\right)}{\sqrt{(1-\rho^2)(1+\sigma^2)}}$.
	\end{enumerate}
\end{proposition}

The model of voluntary referrals generates several predictions regarding worker wages, retention, and the firm's profits. Firstly, the initial wage of a referred worker $j$ is higher than that of a labor market candidate $m$ because the output level of the referring employee is higher than $\bar{\theta}$, as indicated by Lemma \ref{lemma:y_star_existence} and Lemma \ref{lemma:y_tilde_existence}. Moreover, the expected wages of both referred worker $j$ and the worker hired from the labor market $m$ increase in the second period, conditional on their staying in the firm. However, this wage increase is lower for the referred candidate. This result is formally stated in Corollary \ref{cor:wages_vr}.

\begin{corollary}\label{cor:wages_vr}
    In the model where a current employee $i$ who stays with the firm in period $t_i = 2$ is able to refer one of her contacts $j$ for a vacant job position in the firm, the following statements are true:
    \begin{enumerate}[label={\roman*})]
        \item The initial wage of referred candidate $j$ is higher than that of labor market candidate $m$:
        $w_{j,1} \geq w_{m,1}$.
        \item The wage of referred worker $j$ who stayed in the firm in period $t_j = 2$ is higher than his wage in period $t_j = 1$, i.e. $\mathbb{E}[w_{j,2}|y'_j \geq \bar{\theta}] \geq w_{j,1}$, where $y'_j = y_j | y_i$ is an expected output of the referred worker $j$ conditional on the output of the referring employee $y_i$.
        \item The wage of non-referred worker $m$ who stayed in the firm in period $t_m = 2$ is higher than their wage in period $t_m = 1$, i.e. $\mathbb{E}[w_{m,2}|y_m \geq \bar{\theta}] \geq w_{m,1}$.
        \item The difference in wages of referred and non-referred workers decreases over time: $\mathbb{E}[w_{j,2}|y'_j \geq \bar{\theta}] - w_{j,1} \leq \mathbb{E}[w_{m,2}|y_m \geq \bar{\theta}]- w_{m,1}$.
    \end{enumerate}
\end{corollary}

Corollary \ref{cor:wages_vr} is supported by empirical research on referrals. Studies such as \cite{corcoran1980most, korenman1996employment, loury2006some} have shown that wages of referred candidates are higher than those of labor market candidates. Additionally, findings from studies such as \cite{montgomery1991social, simon1992matchmaker, dustmann2016referral} are consistent with the claim that the difference in wages between referred and non-referred candidates decreases over time.

The second result of the model pertains to the difference in retention between referred and non-referred workers. Note that the probability of a worker staying in the firm for her second period depends on her realized output. Due to the correlation between the output of the incumbent worker and her referral, the probability of the referral staying in the second period is higher than that of a labor market candidate. This result is formally stated in Corollary \ref{cor:retention_vr} and is supported by empirical evidence from \cite{simon1992matchmaker, coverdill1998personal, petersen2000offering, kugler2003employee, heath2018firms}.
\begin{corollary}\label{cor:retention_vr}
    In the model where a current employee $i$ who stays with the firm in period $t_i = 2$ is able to refer one of her contacts $j$ for a vacant job position in the firm, the following statement is true:
    \begin{equation}
        P(y_m \geq \bar{\theta}) \leq P(y'_j \geq \bar{\theta}),
    \end{equation}
    where $y'_j = y_j | y_i$ is an output of the referred worker $j$ conditional on the output of the referring employee $y_i$.
\end{corollary}

Another result generated by the model pertains to the relationship between the output of the current employee and the wages and retention of the referred candidate. Since the labor market participants cannot separately observe the general and specific abilities of workers, the only influencing factor in the model is the current employee's output level, denoted by $y_i$. The higher the output of the referring employee, the higher the chances that her social contact $j$ will stay in the same firm for the second period, the higher his initial wage $w_{j,1}$, and his expected wage in the second period. These findings are consistent with research on the role of social networks in the labor market, such as the studies by \cite{saloner1985old, simon1992matchmaker}, and are indirectly supported by empirical evidence in \cite{pallais2016referential, lalanne2016old, levati2020impact}. These results are formally stated in Corollary \ref{cor:relation_current_empl}.
\begin{corollary}\label{cor:relation_current_empl}
    In the model where a current employee $i$ who stays with the firm in period $t_i = 2$ is able to refer one of her contacts $j$ for a vacant job position in the firm, the following statements are true:
    \begin{enumerate}[label={\roman*})]
        \item Initial wage of referred worker $w_{j,1}$ is an increasing function of the referring employee's output level $y_i$.
        \item Expected wage of referred worker $j$ in $t_j = 2$ period,  $\mathbb{E}[w_{j,2}|y_i]$, is an increasing function of the referring employee's output level $y_i$.
        \item Probability of referred worker $j$ to stay in the firm in period $t_j = 2$ is an increasing function of the referring employee's output level $y_i$.
    \end{enumerate}
\end{corollary}

Overall, the predictions of the model with voluntary referrals are consistent with the main empirical findings in the research on referrals that describe labor market outcomes of referred workers and employing firms. 

\subsection{Analysis of the case with ERP}

Now let's return to the question under which conditions an employee referral program is advantageous for the firm. One necessary condition for the firm to benefit from an ERP, which includes a monetary bonus for referring employees whose referrals are hired, is that the firm's threshold for hiring through voluntary referrals, $y^*$, be lower than the employee's threshold, $\tilde{y}$. Otherwise, the introduction of an ERP with a non-negative bonus $b$ will increase the binding constraint $y^*$, because the bonus decreases the expected profit from employing a referred candidate. Consequently, both the expected profit of the firm from referrals and the probability of making referral decrease, making the introduction of an ERP unprofitable for the firm.

To determine whether $\Tilde{y} \geq y^*$, we evaluate the profit difference at $\Tilde{y}$. The profit difference, denoted as $\Delta \Pi_{j,m}(y_i)$ and defined in equation (\ref{eq:profit_dif}), is an increasing function of $y_i$ as demonstrated in the proof of Lemma \ref{lemma:y_star_existence}. Additionally, it is equal to zero when evaluated at $y^*$. Therefore, if $\Delta \Pi_{j,m}(\Tilde{y}) \geq 0$, then the employee's threshold is higher than the firm's threshold, and the necessary condition for an ERP to be profitable is satisfied. Lemma \ref{lemma:erp_existence} presents the main results regarding the dynamics of $\Delta \Pi_{j,m}(\Tilde{y}) \geq 0$:
\begin{lemma}\label{lemma:erp_existence}
    Consider the model where a current employee $i$ who stays with the firm in period $t_i = 2$ is able to refer one of her contacts $j$ for a vacant job position in the firm. Denote the difference between the firm’s expected profit from employing a referred candidate and the expected profit from hiring a labor market candidate, evaluated at $\Tilde{y}$ as $\Delta\Pi_{j,m}(\Tilde{y})$. Then the following statements are true:
    \begin{enumerate}[label={\roman*})]
        \item $\Delta\Pi_{j,m}(\Tilde{y})$ is increasing in $\bar{\theta}$
        \item $\Delta\Pi_{j,m}(\Tilde{y})$ is decreasing in $\rho$ and $\psi_{ij}$.
    \end{enumerate}
\end{lemma}

Lemma \ref{lemma:erp_existence}, together with an additional assumption that there exists some $\theta' \in \bar{\mathbb{R}}$ such that $C(\theta') = 0$, shows that under certain parameter levels of $\bar{\theta}$, $\rho$, and $\psi_{ij}$, the threshold $\Tilde{y}$ can be either greater or lower than $y^*$ \footnote{The additional assumption is necessary to ensure that there are values of $\bar{\theta}$ for which the profit difference evaluated at $\Tilde{y}$ is negative.}. Specifically, when the average general ability required for the position is high, and the correlation between the abilities of referred and referring workers, as well as the social preference parameter of the referring employees, are sufficiently low, then $\Tilde{y} \geq y^*$.

This central result of the model can be interpreted from several viewpoints. Firstly, Lemma \ref{lemma:erp_existence} claims that the firm is more likely to introduce an ERP for "good" jobs as defined by \cite{acemoglu2001good}. In his study, good jobs are high-wage, capital-intensive jobs that require higher labor productivity. This result is also supported by recent empirical evidence in \cite{friebel2023employee}, who claim that an employee referral program is more beneficial when applied to managerial or specialized positions, compared to low-qualified cashier positions in the grocery chain.

% Secondly, the firm is more likely to introduce an ERP if the screening mechanism of job candidates available on the labor market is sufficiently effective, which lowers the variance in the general ability of job candidates. In other words, an ERP is more likely to be introduced for jobs that have clear candidate criteria and in companies that already have elaborate HR systems to search and recruit the best-fit candidates.

Secondly, when a firm implements an ERP, it prioritizes referrals made by current employees using their weak ties. This finding contributes to a large body of research on the impact of strong and weak ties on labor market outcomes, such as the studies cited in \cite{lin1981social, montgomery1992job, montgomery1994weak, granovetter1995getting, yakubovich2005weak, lester2021heterogeneous}.

The intuition behind this result is straightforward. Employees who have strong connections with potential job candidates do not need extra incentives to make referrals. Furthermore, the stronger the connection, the lower the output of the referring employee can be, and the less correlation between workers' abilities is required for the current employee to be motivated to refer her social contact. Consequently, even if these referrals may not necessarily benefit the employer, current employees with "mediocre" abilities and output (i.e., those employees whose output is slightly above the average) might still be willing to refer their friends if they have a high level of the social preference parameter $\psi_{ij}$.

On the other hand, low levels of the social preference parameter make referrals too costly even for high-performing employees whose referrals would be highly beneficial for the firm. Thus, firms are willing to provide additional motivation to these employees to refer their contacts. 

Finally, Lemma \ref{lemma:erp_existence} establishes that several parameters are interdependent and influence the referral benefits of workers and firms, as well as the firm's decision to launch an ERP. Specifically, it sheds light on the controversial empirical results regarding the variation in referral usage across different demographic groups. \cite{holzer1987job, calvo2004effects, loury2006some, pellizzari2010friends,  lalanne2021social} show that workers with lower socioeconomic status and ethnic minorities tend to use referrals more often, yet these groups also tend to have lower average labor market outcomes. The model suggests that various combinations of the strength and quantity of social ties, mean, variance, and correlation between general ability levels within these groups can drive different empirical results.

% There is well-documented evidence of higher usage of referrals by workers with lower socioeconomic status \citep{pellizzari2010friends, corcoran1980most, elliott1999social} and by ethnic minorities \citep{corcoran1980most, datcher1983impact, holzer1987job, green1999racial, loury2006some}. At the same time, \cite{holzer1987job, morrison1990women, calvo2004effects, lalanne2021social, lester2021heterogeneous} show that these groups of workers have lower average labor market outcomes.

For instance, the study in \cite{lester2021heterogeneous} reveals that referrals from family and friends are more frequently used in low-skill job placements, while referrals from business contacts are more often used in high-skill job placements. This result can be interpreted as a trade-off between the general ability levels of the workers and their social preference parameters within the context of the present model. Social networks with low general ability levels and high social preference parameters can generate a high quantity of referrals, but these referrals may not be beneficial for firms. In contrast, social networks with high general ability levels and low social preference parameters generate fewer referrals, but these referrals are highly valued by employers.

The analysis above shows that when a firm observes $\Tilde{y} \geq y^*$, it may consider launching an employee referral program (ERP) with a monetary bonus denoted as $b \geq 0$ for current employees who refer successful candidates. If the firm decides to launch an ERP with bonus $b$, it announces this before observing any job candidates. As a result, the utility of a current employee considering referring her friend changes, and is given by the following equation:
\begin{equation}
        U_{i}(y_i, b) = 
        \begin{cases}
		w_{i,2} + \psi_{ij} w_{j,1}+ b - C(\bar{\theta}) & \text{if } r_i = 2 \\ %+\mathbb{E}[w_{j,2}|y_i]\right)
		w_{i,2} + \psi_{ij} w_{m,1} - C(\bar{\theta}) & \text{if } r_i = 1 \\
        w_{i,2} + \psi_{ij} w_{m,1} & \text{if } r_i = 0
        \end{cases}
\end{equation}
This amended utility function for current employees affects their threshold, which decreases with the bonus $b$:
\begin{equation}\label{eq_tilde_y_b}
    \Tilde{y}(b) = \bar{\theta}+\frac{\left(C(\bar{\theta})-b\right)(1+\sigma^2)}{\psi_{ij}\rho\sigma^2}
\end{equation}
Furthermore, the expected profit for the firm from hiring a referred candidate, denoted as $\Pi_j(y_i,b) = \pi_{j,1}(y_i, b)+ \mathbb{E}[\pi_{j,2}(y_i,b)]$, also decreases due to the bonus $b$:
\begin{equation}
\Pi_j(y_i, b)
= \frac{\rho\left(y_i-\bar{\theta}\right)}{1+\sigma^2}\left(1+\Phi\left(\alpha(y_i)\right)\right)
+ \sqrt{\frac{1-\rho^2}{1+\sigma^2}}\phi\left(\alpha(y_i)\right)-b,
\end{equation}
where $\alpha(y_i) = \frac{\rho\left(y_i - \bar{\theta}\right)}{\sqrt{(1-\rho^2)(1+\sigma^2)}}$. This decrease in the firm's expected profit affects its own threshold, denoted as $y^*(b)$, as well. This threshold is defined in a similar way to the firm's threshold $y^*$ in the case of voluntary referrals and satisfies the following equation: $\Pi_j(y^*(b),b) - \Pi_m = \Delta\Pi_{j,m}\left(y^*(b)\right)-b = 0$. Note that $y^*(b)$ is increasing in $b$. It follows from the fact that $\Delta\Pi_{j,m}(y)$ is an increasing function of $y$\footnote{Which is shown in the Proof of Lemma \ref{lemma:y_star_existence}.} and the higher the bonus $b$, the higher the threshold $y^*(b)$ should be to satisfy the equation $\Delta\Pi_{j,m}\left(y^*(b)\right)=b$.

Hence, under ERP the current employee threshold $\Tilde{y}(b)$ decreases in $b$, while the firm's threshold $y^*(b)$ increases in $b$. Varying the bonus level, the firm can find out the optimal level of the threshold to maximize its profit. Note, that under optimal bonus $b^*$ the firm's threshold cannot be larger than the employee's threshold, i.e. under ERP the following inequality always holds: $\Tilde{y}(b^*) \geq y^*(b^*)$. This happens, because the current employee observes the threshold of the firm and will never refer her friend if her output level $y_i < y^*(b)$. Thus, by decreasing the bonus level until $y^*(b) = \Tilde{y}(b)$ the firm can increase the profit from hiring referred candidate, $\Pi_j(y_i, b)$, while holding probability of referral, $P(y \geq \max \lbrace \Tilde{y}(b), y^*(b)\rbrace)$ constant. 

The overall expected profit of the firm in case of introducing ERP is defined as follows:
\begin{equation}\label{eq:profit_overal_erp}
    \Pi(b) = P\bigl(y_i \geq \Tilde{y}(b)\bigr) \mathbb{E}[\Pi_j(y_i, b)| y_i \geq \Tilde{y}(b)] + \Bigl(1-P\bigl(y_i \geq \Tilde{y}(b)\bigr)\Bigr)\Pi_m
\end{equation}
The equilibrium behavior of the model with an Employee Referral Program (ERP) is formally stated in Proposition \ref{prop:eq_erp}:
\begin{proposition}\label{prop:eq_erp}
    In the model where the firm is able to introduce an employee referral program with monetary bonus $b$ and a current employee i who stays with the firm in period $t_i = 2$ is able to refer one of her contacts $j$ for a vacant job position in the firm, the current employee’s referral decision, the firm’s optimal bonus, hiring and wage decisions, and its profits are determined as follows:
    \begin{enumerate}[label={\roman*})]
        \item In the beginning of period $t_j = 1$ the firm announces the launch of the ERP with a bonus $b^* \geq 0$, which is paid in case the referred candidate $j$ is hired, if $\Tilde{y} \geq y^*$. The optimal bonus is determined as $b^*=\arg\max_{b}\Pi(b)$. If $\Tilde{y} < y^*$, the firm does not announce the ERP, and the bonus is assumed to be equal to $b^* = 0$.
		\item The firm hires referred worker $j$ in $t_j = 1$ and pays wage $w_{j,1} = \bar{\theta}+\frac{\rho \sigma^2}{1+\sigma^2} \left(y_i - \bar{\theta}\right)$ if the referring worker's output level $y_i \geq \Tilde{y}(b^*)$. Otherwise, the firm hires the labor market candidate $m$ and pays them $w_{m,1} = \bar{\theta}$. The expected profit of the firm from hiring referral $j$ by the current employee $i$ with output level $y_i$ in $t_j = 1$ is equal to $\pi_{j,1}(b^*) = \frac{\rho}{1+\sigma^2}\left(y_i - \bar{\theta}\right)-b^*$.
        \item In period $t_j = 2$ the firm retains referred worker $j$ and pays wage $w_{j,2} = \bar{\theta} + \frac{\sigma^2}{1+\sigma^2}\left( y_j - \bar{\theta} \right)$ if $y_j \geq \bar{\theta}$. Otherwise, worker $j$ leaves the firm and accepts the outside offer with the wage $w_{j,2}$; the firm hires the labor market candidate and pays them $w_{m,1} = \bar{\theta}$. The expected profit of the firm in $t_j = 2$ is equal to $\Pi_{j,2} = \frac{\rho\left(y_i-\bar{\theta}\right)}{1+\sigma^2}\Phi(\alpha(y_i))
        +\sqrt{\frac{1-\rho^2}{1+\sigma^2}}\phi(\alpha(y_i))$, where $\alpha (y_i) = \frac{\rho\left(y_i - \bar{\theta}\right)}{\sqrt{(1-\rho^2)(1+\sigma^2)}}$.
	\end{enumerate}
\end{proposition}

The model of the firm's decision to launch an employee referral program generates predictions about labor market outcomes for workers and employers, which are supported by empirical evidence. The model predicts that if the employee's threshold output level for a referral ($\Tilde{y}$) is greater than or equal to the employer's threshold output level ($y^*$) and the firm considers implementing an employee referral program (ERP),  then the probability of a referral increases with the bonus, while the average expected output of referred workers decreases. This is formally stated in the following Corollary:

\begin{corollary}\label{cor:erp_emp_evidence}
In the model where $\Tilde{y} \geq y^*$ and the firm introduces an employee referral program with a monetary bonus $b \geq 0$, the following statements are true:
\begin{enumerate}[label = {\roman*)}]
\item The probability that a current employee $i$ will refer a friend $j$ is increasing in the level of the bonus $b$, i.e., $P(y_i\geq \Tilde{y}(b))$ is an increasing function of $b$.
\item The average expected output of the referred worker $j$ is decreasing in the level of the bonus $b$, i.e., $\mathbb{E}[y_j|y_i\geq \Tilde{y_i}(b)]$ is a decreasing function of $b$.
\item The average wage of the referred worker $j$ in their first period $t_j = 1$ is decreasing in the level of the bonus $b$, i.e., $\mathbb{E}[w_{j,1}|y_i\geq \Tilde{y_i}(b)]$ is a decreasing function of $b$.
\end{enumerate}
\end{corollary}

The first two statements of Corollary \ref{cor:erp_emp_evidence} align with recent research findings in \cite{friebel2023employee}, which show that increasing referral bonuses leads to more referrals and higher-quality referrals compared to non-referrals. However, as referral bonuses continue to increase, the quality of referrals decreases. 

Furthermore, the current model provides additional support for the finding in \cite{friebel2023employee} that implementing an ERP can increase firm profits, provided the referral bonus is not too large. The model shows that the firm can extract profits from voluntary referrals, as stated in Lemma \ref{lemma:erp_existence}. Under certain conditions, the firm can also benefit further from the introduction of an ERP. Specifically, the benefits from an ERP are greatest in labor markets with high expected general worker ability, an efficient screening mechanism for job candidates, and weak ties between workers.

% This result is formally stated in Corollary \ref{cor:erp_optimal_bonus}:
% \begin{corollary}\label{cor:erp_optimal_bonus}
%     In the model where the employee's threshold output level for a referral $\Tilde{y}$ is greater than or equal to the employer's threshold output level $y^*$ and the firm introduces an employee referral program (ERP), the optimal bonus level $b^*=\arg\max_{b}\Pi(b)$ exhibits the following properties:
%     \begin{enumerate}[label = {\roman*)}]
%         \item $b^*$ is a non-increasing function of $\rho$ and $\psi_{ij}$, indicating that as the strength of social ties between referred candidates and referrers increases, the optimal bonus level decreases.
%         \item $b^*$ is a non-increasing function of $\bar{\theta}$, indicating that as the average ability level of referred candidates increases, the optimal bonus level also increases.
%     \end{enumerate}
% \end{corollary}



\section{Extension: Observing Firm-Specific Ability} \label{sec:extension}


This section presents the model with the amended Assumption A5, which may initially appear restrictive as it does not allow labor market participants to distinguish between the general and firm-specific ability of the worker. However, the alternative assumption allows all market participants to separately observe both the general and firm-specific ability levels of workers.

\begin{enumerate}[label={A5.}{\arabic*}.]
    \item When a worker enters the labor market at the beginning of their career, $t_l =1$, none of the market participants, including the worker, have direct knowledge of the true values of the worker's general ability, denoted by $\theta_l$, and their firm-specific ability, denoted by $\mu_l$. However, they share common prior knowledge that $\theta_l$ and $\mu_l$ are independently and normally distributed with $\theta_l \sim \mathcal{N}\left( \bar{\theta}, \sigma^2\right)$ and $\mu_l \sim \mathcal{N}\left( 0, 1\right)$, where $\bar{\theta} \geq 0$ and $\sigma^2 \in (0, \infty)$. All labor market participants can observe realized values of both the worker's general ability, $\theta_l$, and firm-specific ability, $\mu_l$, and the worker output, $y_l$ at the end of the first period $t_l = 1$.
\end{enumerate}

This assumption can be interpreted in several ways. First, it applies to specific jobs where the worker's output does not directly depend on the job design or the team they work with. Examples include freelance jobs and jobs outsourced by the firm to third parties. Another case is when workers are allowed to change employers, and thus, the market's belief about the worker's general ability level becomes more precise based on their career paths with different employers. In fact, Assumptions A5 and A5.1 can be interpreted as two different ends of a spectrum, where labor market participants can distinguish between the general and firm-specific ability of workers to varying degrees.

Assumption A5.1 has an impact on the outcomes of labor market participants. The wage of labor market candidate $m$ in their first period remains the same: $w'_{m,1} = \bar{\theta}$. However, in their second period, the wage of worker $m$ is now equal to $w'_{m,2} = \theta_m$. The wage of the worker $j$, referred by the current employee $i$ with $\theta_i$ and $\mu_i$, in the first period is given by $w'_{j,1} = \mathbb{E}[\theta_j | \theta_i] = \bar{\theta} + \rho(\theta_i - \bar{\theta})$. Additionally, the wage of the referred worker $j$ in the second period is $w'_{j,2} = \theta_j$.

The firm's profit is also affected by these changes. The profit from hiring labor market participant $m$ in the first period is zero ($\pi'_{m,1} = 0$), while the profit in the second period is $\pi'_{m,2} = \mu_m$. On the other hand, the profit in the first period from employing the referred candidate $j$ is $\pi'_{j,1} = \mathbb{E}[\mu_j|\mu_i] = \rho\mu_i$, and the profit in the second period from employing the referred candidate is $\pi'_{j,2} = \mu_j$.

\subsection{Analysis of the case without referrals}

The equilibrium behavior of the model without referrals is formally stated in Proposition \ref{prop:ext_eq_nr}:
\begin{proposition}\label{prop:ext_eq_nr}
     Consider the model where all market participants can observe both the general and firm-specific ability of workers. In the absence of referrals by current employees, the firm's hiring and wage decisions, and its profits are determined as follows:
    \begin{enumerate}[label={\roman*})]
        \item At the start of period $t_m = 1$, the firm hires a worker $m$ from the labor market and pays them a wage of $w'_{m,1} = \bar{\theta}$. The expected profit of the firm in period $t_m = 1$ from hiring worker $m$ is zero.
        \item At the start of period $t_m = 2$, the firm either retains worker $m$ and pays them a wage of $w'_{m,2} = \theta_m$, if their firm-specific ability level $\mu_m$ in period $t_m = 1$ is greater than or equal to $0$, or lets worker $m$ leave to accept an outside offer of $w'_{m,2}$, if $\mu_m < 0$, while the firm hires another labor market candidate $m'$ and pays them $w'_{m,1} = \bar{\theta}$. The expected profit of the firm in period $t_m = 2$ is $\Pi'_{m,2} = \phi(0)$.
	\end{enumerate}
\end{proposition}

Under Assumption A5.1, the profit of the firm is no longer dependent on the general ability level of the worker. Instead, the firm's profit is influenced solely by the worker's expected firm-specific ability. The firm retains workers who have a non-negative firm-specific ability ($\mu_l \geq 0$), meaning they are a good fit for the firm.

If a worker $m$ is found to have low general ability ($\theta_m < \bar{\theta}$) but a good match ($\mu_m \geq 0$), the firm adjusts their wage based on the general ability level and retains them. Interestingly, high-ability workers who do not match well with the firm choose to leave after the first period and accept offers from other firms in the labor market.

It is important to note that under Assumption A5.1, a worker's wage does not necessarily increase with tenure. The wage in the second period ($w'_{j,2} = \theta_j$) can be either higher or lower than the wage in the first period ($w'_{j,1} = \bar{\theta}$). Wage increases occur only for workers whose general ability level is above average. Although this result differs from the implications of the initial model under Assumption A5, it is partially supported by the labor market literature. \cite{gibbons1999theory} present a theoretical model that allows for real-wage decreases, while empirical studies by \cite{mclaughlin1994rigid}, \cite{baker1994internal, baker1994wage}, and \cite{card1997does} provide evidence of real-wage decreases in firms.

\subsection{Analysis of the case with voluntary referrals}

The equilibrium behavior of the model under voluntary referrals also changes. Firstly, the employer's constraint for hiring referred candidates is no longer dependent on realization of the general specific ability of the referring employee, $\theta_i$. Instead, this threshold is determined by the level of firm-specific ability of the referring candidate, $\mu^*$. This result is formally presented in Lemma $\ref{lemma:mu_star_existence}$.
\begin{lemma}\label{lemma:mu_star_existence}
    Consider the model where all market participants can observe both the general and firm-specific ability of workers and a current employee $i$ who stays with the firm in period $t_i = 2$ is able to refer one of her contacts $j$ for a vacant job position in the firm. The firm will hire the referral $j$ only if the current employee's firm-specific ability is greater than or equal to a certain threshold: $\mu_i \geq \mu^*$. This threshold is greater than zero ($\mu^* > 0$) and is determined by solving the following equation:
    \begin{equation}
        \Delta\Pi'_{j,m}(\mu^*) = 0,
    \end{equation}
    where $\Delta\Pi'_{j,m}(\mu_i) = \rho\mu_i\left(1 + \Phi\Bigl( \frac{\rho\mu_i}{\sqrt{1-\rho^2}} \Bigr) \right) + \sqrt{1-\rho^2}\phi \Bigl( \frac{\rho\mu_i}{\sqrt{1-\rho^2}} \Bigr) - \phi(0)$.
\end{lemma}

Lemma \ref{lemma:mu_star_existence} states that the firm benefits from candidates who are referred by current employees with a firm-specific ability, $\mu_i$, higher than or equal to $\mu^*$. On the other hand, the employee constraint is not influenced by the realization of firm-specific ability, $\mu_i$, but rather depends on the general ability level, $\theta_i$. Specifically, the difference in utilities for the current employee when deciding to make a referral is now given by $\Delta U'_{i}(\theta_i) = \psi_{ij}\rho(\theta_i-\bar{\theta})- C(\bar{\theta})$. The threshold dynamics is similar to the initial case and is formally presented in Lemma \ref{lemma:theta_star_existence}.
\begin{lemma}\label{lemma:theta_star_existence}
    Consider the model where all market participants can observe both the general and firm-specific ability of workers and a current employee $i$ who stays with the firm in period $t_i = 2$ is able to refer one of her contacts $j$ for a vacant job position in the firm.  The current employee will never make a referral if her social preference parameter $\psi_{ij}$ is equal to zero. However, if $\psi_{ij}$ is positive, the current employee will make a referral only if her general ability level is greater than or equal to the following threshold: $\theta_i \geq \theta^*$, where $\theta^* = \bar{\theta}+\frac{C(\bar{\theta})}{\psi_{ij}\rho}$.
\end{lemma}

Note that the underlying mechanism of referrals remains unchanged under Assumption A5.1. Lemma \ref{lemma:theta_star_existence} demonstrates that the social preferences of current employees towards their social contacts drive voluntary referrals in the labor market. This leads to a similar result as the one stated in Lemma \ref{lemma:y_tilde_existence}. The difference lies in the realization of this mechanism. The firm's threshold $\mu^*$ is still observed by the referring employee, and thus she will not make a referral if $\mu_i < \mu^*$. However, the independence between general ability and firm-specific ability means that in order to make a referral, the current employee's characteristics have to satisfy both conditions simultaneously, resulting in the probability of making a referral being equal to $P(r_i = 2) = P(\theta_i \geq \theta^*)P(\mu_i \geq \mu^*)$. The equilibrium behavior of the model presented in Proposition \ref{prop:ext_eq_vr}.
\begin{proposition}\label{prop:ext_eq_vr}
    In the model where all market participants can observe both the general and firm-specific ability of workers and a current employee $i$ who stays with the firm in period $t_i = 2$ is able to refer one of her contacts $j$ for a vacant job position in the firm, the current employee's referral decision, the firm’s hiring and wage decisions, and its profits are determined as follows:
    \begin{enumerate}[label={\roman*})]
		\item At the start of period $t_j = 1$ the firm hires referred worker $j$ and pays him wage $w'_{j,1} = \bar{\theta}+\rho\left(\theta_i - \bar{\theta}\right)$ if the general ability level of the referring employee $\theta_i \geq \theta^*$ and the firm-specific ability level of the referring employee $\mu_i \geq \mu^*$. Otherwise, the firm hires the labor market candidate $m$ and pays them $w'_{m,1} = \bar{\theta}$. The expected profit of the firm from hiring referral $j$ by the current employee $i$ in $t_j = 1$ is equal to $\pi'_{j,1} = \rho\mu_i$.
        \item At the start of period $t_j = 2$ the firm retains referred worker $j$ and pays him wage $w'_{j,2} = \theta_j$ if $\mu_j \geq 0$. Otherwise, the worker $j$ leaves the firm and accepts the outside offer with the wage $w'_{j,2}$; the firm hires the labor market candidate and pays them $w'_{m,1} = \bar{\theta}$. The expected profit of the firm from the worker $j$ referred by the current employee $i$ in $t_j = 2$ is equal to $\Pi'_{j,2} = \rho\mu_i \Phi \Bigl(\frac{\rho\mu_i}{\sqrt{1-\rho^2}} \Bigr) + \sqrt{1-\rho^2}\phi \Bigl(\frac{\rho\mu_i}{\sqrt{1-\rho^2}} \Bigr)$.
	\end{enumerate}
\end{proposition}

Most of the predictions from the initial model, as stated in Corollaries \ref{cor:wages_vr}, \ref{cor:retention_vr}, and \ref{cor:relation_current_empl}, also hold in the amended model. The initial wage of the referred candidate $j$ is higher than that of the labor market candidate $m$, as $w'_{j,1} = \bar{\theta} + \rho(\theta_i - \bar{\theta}) \geq \bar{\theta} = w'_{m,1}$ due to the constraints faced by the referring employee ($\theta_i \geq \theta^* > \bar{\theta}$). The probability of the referred worker staying in the firm in the second period is higher than that of the non-referred worker, because $P(j\text{ stays in }t_j=2) = \Phi\Bigl( \frac{\rho \mu_i}{\sqrt{1-\rho^2}} \Bigr) > \Phi(0) = P(m\text{ stays in }t_m=2)$, where $\mu_i \geq \mu^* > 0$. Furthermore, the higher the general ability of the referring worker, the higher the initial wage of the referred candidate, while the probability of the referred candidate staying in the firm increases with the firm-specific ability level of the referring employee, supporting the claims in Corollary \ref{cor:relation_current_empl}.

However, the amended model allows for testable hypotheses regarding the benefits that the firm obtains from voluntary referrals. The profit of the firm from hiring candidate $j$ referred by the current employee $i$ with ability levels $\theta_i$ and $\mu_i$, denoted as $\Pi'_j(\mu_i)$, is given by:
\begin{equation}
    \Pi'_j(\mu_i) = 
    \rho\mu_i
    \left(
    1 + \Phi\biggl( \frac{\rho\mu_i}{\sqrt{1-\rho^2}} \biggr) 
    \right) 
    + 
    \sqrt{1-\rho^2}\phi \biggl( \frac{\rho\mu_i}{\sqrt{1-\rho^2}} \biggr)
\end{equation}
Thus, the expected profit of the firm from filling an open job position when voluntary referrals are allowed is given by $\Pi'(VR) = P(r_i = 2)\mathbb{E}[\Pi'_j(\mu_i)|\mu_i \geq \mu^*] + P(r_i = 0)\Pi'(NR)$, where $\Pi'(NR) = \phi(0)$ represents the expected profit of the firm from hiring a labor market candidate. Simplifying further, the difference between the expected profit of the firm under voluntary referrals and without referrals can be expressed as follows:
\begin{equation}\label{eq:profit_vr_amended}
    \Pi'(VR) - \Pi'(NR) = 
    \Biggl( 
    1 - \Phi\biggl( \frac{C(\bar{\theta})}{\rho \psi_{ij} \sigma} \biggr)
    \Biggr)
    \Biggl( 
    \int^\infty_{\mu^*}\Pi'_j(t)\phi(t)dt 
    -
    \Bigl( 1 - \Phi (\mu^*)\Bigr) \phi(0)
    \Biggr)
\end{equation}

The difference between the expected profit of the firm under voluntary referrals and without referrals represents the additional benefits the firm gains from voluntary referrals by current employees. It is worth noting that the second factor on the right-hand side of equation (\ref{eq:profit_vr_amended}) is independent of the mean, $\bar{\theta}$, and standard deviation, $\sigma$, of the general ability, as well as the social preference parameter of the current employee towards her friend, $\psi_{ij}$. Furthermore, it is always non-negative\footnote{This statement is shown in the proof of Corollary \ref{cor:profit_vr_amended} in Appendix A}. Therefore, the firm's benefits from voluntary referrals are greater when the candidates' general ability is low and uncertain, and when the employees' social preferences towards their friends are strong (i.e., the social ties are strong). This result is formally stated in Corollary \ref{cor:profit_vr_amended}.
\begin{corollary}\label{cor:profit_vr_amended}
   Consider the model where all market participants can observe both the general and firm-specific ability of workers and a current employee $i$ who stays with the firm in period $t_i = 2$ is able to refer one of her contacts $j$ for a vacant job position in the firm. Denote the difference between the firm’s expected profit from filling an open job position when voluntary referrals are allowed and the expected profit from hiring a labor market candidate as $\Pi'(VR) - \Pi'(NR)$. Then the following statements are true:
    \begin{enumerate}[label={\roman*})]
        \item $\Pi'(VR) - \Pi'(NR)$ is decreasing in $\bar{\theta}$
        \item $\Pi'(VR) - \Pi'(NR)$ is increasing in $\sigma$ and $\psi_{ij}$.
    \end{enumerate}
\end{corollary}

Corollary \ref{cor:profit_vr_amended} provides valuable insights into the benefits that firms derive from voluntary referrals. One important implication is the explanation for the higher usage of referrals by smaller firms compared to larger organizations \citep{marsden2001social}. Smaller firms often lack extensive HR departments, resulting in less precise screening mechanisms and recruiting practices. In such cases, voluntary referrals serve as a valuable source of information. However, as formal hiring mechanisms become more effective, the additional information provided by current employees about referred candidates becomes less significant. This leads to a decrease in the gains from referrals and, consequently, an increase in the employee's threshold $\theta^*$. As a result, fewer employees are motivated to refer their friends for job positions. 

It is important to note that in the amended model, the firm's profit from hiring a referred candidate, $\Pi'_j(\mu_i)$, is independent of parameters such as $\sigma$, $\bar{\theta}$, and $\psi_{ij}$. Therefore, the increased benefits from voluntary referrals primarily result from an increased number of referrals for higher values of $\sigma$ and $\psi_{ij}$. On the other hand, in the initial model, the firm's profit from hiring a referred candidate, $\Pi_j(y_i)$, decreases as $\sigma$ increases, indicating that the expected profit from each individual referral is higher for firms with well-developed recruiting practices.

It is important to highlight that the difference between the initial and amended models lies in the assumption regarding the ability of labor market participants to distinguish between general and firm-specific abilities of workers, making these two models represent opposite ends of the same spectrum. Therefore, assuming that labor market participants can distinguish between these two types of ability only to a certain extent leads to the following conclusion: young firms with developing recruiting practices rely more on voluntary referrals than their larger counterparts with elaborated HR departments, but their expected profit from each individual referral is lower than that of larger firms. This conclusion is supported by empirical evidence presented in \cite{black2020network}.

\subsection{Analysis of the case with ERP}

Given that the amended model incorporates two distinct constraints that affect the number of referrals and the expected firm-specific ability of referred workers, the firm may have an interest in implementing external incentives to encourage current employees to increase their referrals.

The introduction of an ERP with a monetary bonus $b$ for successful referrals influences these two constraints as follows. The employee's threshold, $\theta^*(b) = \bar{\theta} + \frac{C(\bar{\theta}) - b}{\psi_{ij}\rho}$, decreases with increasing bonus level $b$. On the other hand, the firm's threshold, denoted as $\mu^*(b)$, increases with the bonus level. Consequently, the ERP increases the number of current employees who are willing to refer their friends. This is reflected in the increasing probability $P(\theta_i \geq \theta^*(b)) = 1 - \Phi\left(\frac{C(\bar{\theta}) - b}{\psi_{ij} \rho \sigma}\right)$ with respect to $b$. However, the ERP also excludes some current employees with low levels of $\mu_i$ who would otherwise be willing to refer their friends, as the threshold $\mu^*(b)$ increases. This is reflected in the decreasing probability $P(\mu_i \geq \mu^*(b)) = 1 - \Phi\left(\mu^*(b)\right)$ with respect to $b$.

Therefore, the firm would be interested in implementing an ERP with a monetary bonus $b\geq 0$ when the constraint faced by current employees is relatively stronger compared to the firm's constraint. In essence, the firm introduces an ERP in response to under-referral from the employees, similar to the initial model. However, for the ERP to be effective, the positive impact of the program needs to outweigh the negative effects stemming from the increase in the firm's threshold $\mu^*(b)$ and the additional costs incurred for successful referrals. In other words, the difference between the firm's overall expected profits with and without the ERP, denoted as $\Delta\Pi'(b) = \Pi'(b) - \Pi'(VR)$, should be positive: $\Delta\Pi'(b) \geq 0$. Here, $\Pi'(b)$ represents the firm's overall expected profits when the ERP is implemented with a bonus $b$, and it can be calculated as follows:
\begin{equation}
    \Pi'(b) = 
    \Biggl( 
    1 - \Phi\biggl( \frac{C(\bar{\theta})-b}{\rho \psi_{ij} \sigma} \biggr)
    \Biggr)
    \Biggl( 
    \int^\infty_{\mu^*(b)}\Pi'_j(t)\phi(t)dt 
    -
    \Bigl( 1 - \Phi \bigl(\mu^*(b)\bigr)\Bigr) \Bigl(\phi(0) + b\Bigr)
    \Biggr)
    + \phi(0)
\end{equation}

The equilibrium behavior of the amended model with an ERP is formally stated in Proposition \ref{prop:ext_eq_erp}:
\begin{proposition}\label{prop:ext_eq_erp}
    Consider the model where all market participants can observe both the general and firm-specific ability of workers. When the firm is able to introduce an employee referral program with monetary bonus $b$ and a current employee i who stays with the firm in period $t_i = 2$ is able to refer one of her contacts $j$ for a vacant job position in the firm, the current employee’s referral decision, the firm’s optimal bonus, hiring and wage decisions, and its profits are determined as follows:
    \begin{enumerate}[label={\roman*})]
        \item In the beginning of period $t_j = 1$ the firm announces the launch of the ERP with a bonus $\Tilde{b} \geq 0$, which is paid in case the referred candidate $j$ is hired. The optimal bonus is determined as $\Tilde{b}=\max\{0,\arg\max_{b}\Pi'(b)\}$.
		\item The firm hires referred worker $j$ and pays him wage $w'_{j,1} = \bar{\theta}+\rho\left(\theta_i - \bar{\theta}\right)$ if the general ability level of the referring employee $\theta_i \geq \theta^*(\Tilde{b})$ and the firm-specific ability level of the referring employee $\mu_i \geq \mu^*(\Tilde{b})$. Otherwise, the firm hires the labor market candidate $m$ and pays them $w'_{m,1} = \bar{\theta}$. The expected profit of the firm from hiring referral $j$ by the current employee $i$ in $t_j = 1$ is equal to $\pi'_{j,1}(\Tilde{b}) = \rho\mu_i-\Tilde{b}$.
        \item  At the start of period $t_j = 2$ the firm retains referred worker $j$ and pays him wage $w'_{j,2} = \theta_j$ if $\mu_j \geq 0$. Otherwise, the worker $j$ leaves the firm and accepts the outside offer with the wage $w'_{j,2}$; the firm hires the labor market candidate and pays them $w'_{m,1} = \bar{\theta}$. The expected profit of the firm from the worker $j$ referred by the current employee $i$ in $t_j = 2$ is equal to $\Pi'_{j,2} = \rho\mu_i \Phi \Bigl(\frac{\rho\mu_i}{\sqrt{1-\rho^2}} \Bigr) + \sqrt{1-\rho^2}\phi \Bigl(\frac{\rho\mu_i}{\sqrt{1-\rho^2}} \Bigr)$.
	\end{enumerate}
\end{proposition}

The amended model predicts that if the firm introduces an ERP with a positive bonus $b$, then the expected general ability of referred workers decreases, while the expected firm-specific ability of referred workers increases. This is due to the relationship between the bonus $b$ and the optimal values of $\theta^*(b)$ and $\mu^*(b)$. Specifically, $\theta^*(b)$ is decreasing in $b$, meaning that higher bonuses lead to lower expectations of general ability. On the other hand, $\mu^*(b)$ is increasing in $b$, indicating that higher bonuses are associated with higher expectations of firm-specific ability.

\section{Discussion} \label{sec:discussion}

The current model is based on several fundamental assumptions that ensure the generalizability of the results and predictions generated by the model. Firstly, the labor market model shares similarities with those described in \cite{gibbons1999theory} and \cite{ekinci2016employee} in terms of the firm's lack of market power in the labor market, the form of the production function, and the timing of the model. 

The idea of two different types of worker abilities (general ability and specific ability) can be traced back to the seminal works of \cite{becker1962investment, becker1975investment} and \cite{jovanovic1979job}, but is not necessarily constrained by the definition of firm-specific capital, which has come under criticism in recent years  \citep{gibbons2004task, gathmann2010general}. In contrast, specific ability can be interpreted as the worker's productivity loss associated with changing the employers. This may be caused by differences in the weights of skills used for similar jobs in different firms \citep{lazear2009firm}, or the magnitude of differences between the tasks in one job compared to another \citep{gibbons2004task}. 

The main mechanism of referrals is based on the assumptions of assortative matching by \cite{montgomery1991social}, together with the assumptions of the model discussed in \cite{friebel2023employee}. The assumption in \cite{friebel2023employee} about current employees' altruism towards their social contacts constitutes the main driving force of both voluntary referrals and referrals under ERP in the model. 

Another core assumption of the model is the symmetric information available to the firm and the referring employee. This assumption is different from most of the theoretical studies on referrals, which assume asymmetric information of the current employee, who observes (to some extent) the ability (or match) of the referred employee, while the firm and other market participants do not have access to this information ex ante \citep{saloner1985old, beaman2012gets, ekinci2016employee}. The current model also considers referrals as a mechanism for eliciting information about the firm-worker specific match through the network of social contacts. However, in this model, the act of a current employee making a referral does not provide any additional information to the employer unless the referrer's abilities and output are known.

This assumption introduces the concept of assortative matching to the model in a slightly different manner while keeping it simple. Another advantage of this approach is the ability to vary the "strength" of assortative matching within the social network, providing a useful tool for further research on the relationship between the strength of social ties within social networks and labor market outcomes of referrals. Thus, it contributes to the body of literature on variation in the usage of referrals across different demographic groups, such as \cite{montgomery1994weak, granovetter1995getting, calvo2004effects, kuzubas2009endogenous, lester2021heterogeneous}, etc.

Assumption (A5) states that labor market participants observe only a worker's output, rather than both the general and specific abilities separately. The rationale behind this assumption is that other firms are unable to accurately assess the worker's contribution to the firm's output. While this assumption may appear restrictive, an alternative assumption (A5.1) is considered, which allows all market participants to observe both the general and specific abilities. However, this assumption applies only to specific jobs where the worker's output is not directly influenced by job design or the team they work with, such as freelance jobs or jobs outsourced to third parties.

The analysis of the amended model with Assumption (A5.1) reveals several insights. First, the wages of referred workers remain higher than those of labor market candidates, and their turnover is lower. Additionally, candidates referred by high-ability employees have a higher likelihood of staying in the firm and receiving higher wages. Furthermore, the benefits derived by the firm from voluntary referrals are greater when the general ability of candidates is low and uncertain, and when employees and referred workers share strong social ties.

The model generates several empirical predictions. Firstly, it identifies the main drivers of voluntary referrals in the firm, namely the social preferences of current employees towards their friends and the correlation in abilities (both general and specific) of the contacts (as demonstrated in Lemma \ref{lemma:y_tilde_existence}). The inclusion of voluntary referrals in the model not only contributes to the literature on employee referral programs but also sheds light on a broad range of research on job referrals (a comprehensive literature review is provided by \cite{topa2011labor}). Unlike most theoretical papers that explain the behavior of firms and employees under ERP \citep{beaman2012gets, ekinci2016employee}, the equilibrium behavior in the current model indicates that current employees do refer their friends under certain labor market conditions, even in the absence of external incentives. This result is consistent with evidence from \cite{holzer1987hiring, granovetter1995getting, pellizzari2010friends, lester2021heterogeneous}. Moreover, the field experiment conducted by \cite{heath2018firms} in Bangladeshi garment factories showed that under specific circumstances, current employees are willing to forgo some of their own wages to refer their friends, which is also in line with the findings of the model.

Secondly, the model identifies the necessary conditions for a firm to implement an employee referral program that includes a fixed material bonus paid to the referring employee if her referral is hired (see Lemma \ref{lemma:y_tilde_existence} and Lemma \ref{lemma:erp_existence}). Specifically, the model shows that the ERP is more beneficial for the firm when the job requires high qualifications and when the ties between the referring and referred workers are weak. In contrast, voluntary employee referrals are prevalent for the jobs with low qualifications and network of workers with strong social ties. Additionally, the model sheds light on why firms use fixed payments rather than bonuses contingent on referral performance to incentivize current employees. Lemma \ref{lemma:y_tilde_existence} indicates that when $\Tilde{y} \geq y^*$, the firm faces under-referral from the employee's side; that is, some workers have output levels high enough to generate referrals that would be profitable for the firm, but not beneficial enough for them. Thus, the purpose of the ERP is to motivate medium-ability workers to make referrals. However, high-ability workers have output levels high enough to generate referrals even without the material bonus. Therefore, the firm faces a dilemma: it doesn't need to incentivize high-profile employees with the most profitable referrals, but wants to motivate employees who generate referrals with lower profitability for the firm due to their lower output levels. Introducing the ERP exclusively for workers with slightly higher-than-average output  may disincentivize high-ability employees from making referrals voluntarily.

Finally, the model generates several predictions that align with much of the existing research on labor market referrals, thereby enhancing the validity of the model. Specifically, it shows that the initial wage of referred workers is higher than that of labor market participants, and that the expected wages of both referred and non-referred workers increase over time, although the wage increase is lower for referred workers (see Corollary \ref{cor:wages_vr}). These findings are supported by studies conducted by \cite{corcoran1980most}, \cite{montgomery1991social}, \cite{dustmann2016referral}, and others. Additionally, the model predicts that the turnover of referred workers is lower than that of non-referred workers (see Corollary \ref{cor:retention_vr}), which is consistent with empirical evidence from \cite{pallais2016referential}, \cite{lalanne2016old}, and \cite{lalanne2021social}.

Furthermore, the model reveals a positive relation between the productivity (and thus wage and tenure) of the referring employee and the outcomes of the referred candidate. In particular, the higher the output of the referring employee, the higher the initial wage of the referred worker, and the lower the probability of the referred worker to leave the firm after the first period (see Corollary \ref{cor:relation_current_empl}). These claims are partially supported by evidence from \cite{simon1992matchmaker}, \cite{kugler2003employee}, \cite{pallais2016referential}, and \cite{levati2020impact}, which demonstrates the positive effect of old boy networks on the labor market outcomes of referred workers. However, to my knowledge, empirical investigations establishing a causal relationship between the wage dynamics of referred and referring workers have not been conducted yet, and this remains the subject of further theoretical and empirical research.

Unlike most of studies on referrals this paper provides a model that considers both voluntary referrals and an employee referral program. The model predicts that under an ERP, the probability of a referral increases with the bonus level, while the average expected output of referred workers decreases. These findings are supported by the research of \cite{friebel2023employee}, which demonstrates that increasing referral bonus leads to higher quantity but lower quality of referrals.

Moreover, the model allows for an investigation into the dynamics of benefits for both the firm and the worker arising from referrals. Special cases of the baseline and extended models, showcasing changes in the probability of referrals, firm profits, and the optimal bonus level, are presented in Appendix B and Appendix C.

Although the model explains most of the empirical findings about the effect of referrals on labor market outcomes, it has several limitations that could serve as a starting point for further research. First, the model does not differentiate between two types of voluntary referrals: formal referrals, which are known to the firm, and informal job referrals, where a current employee suggests their social contacts to apply for open positions without the firm's knowledge. Introducing this distinction in the information structure of the model could potentially impact the equilibrium behavior of labor market participants in the case of informal voluntary referrals and may necessitate modifications to the presented results. Exploring this direction of research could help explain the differences in findings between studies that utilize macro data from aggregated datasets (such as NLSY and PSID) and those based on microeconomic data at the firm level. 

Other potential directions for further research include investigating alternative mechanisms for incentivizing referring employees, exploring the impact of different information structures on labor market conditions, and internalizing social preference parameters and ability correlations to examine principal-agent and moral hazard problems within the context of employee referrals.

\section{Conclusion} \label{sec:conclusion}

This paper develops a model of employee referrals to examine the conditions on the labor market under which the implementation of an employee referral program (ERP) is beneficial for firms. Unlike most existing research on labor market referrals, the current model considers both voluntary referrals by current employees and the implementation of an ERP that includes a fixed material bonus paid to the referring employee if their referral is hired. In order to do this, the model is based on the following assumptions:
\begin{itemize}
    \item There are two types of worker ability: general ability and specific firm-worker ability \citep{becker1962investment, becker1975investment, gibbons2004task, lazear2009firm}. 
    \item Current employees have social preferences towards their social contacts \citep{bandiera2009social, friebel2023employee}.
    \item Ability levels are positively correlated among connected workers \citep{montgomery1991social}.
\end{itemize}

The analysis provides insights into the effectiveness of referrals under different labor market conditions. The core idea of the model is that when current employees make a referral, they do not convey any additional information to their employer beyond their own ability levels and the fact of social connection with the job applicant. The model predicts that some current employees are willing to voluntarily refer their contacts even without external incentives from their employer. However, if the job requires highly qualified workers while the social ties among them are weak, the firm may experience a situation of under-referral, wherein some potentially beneficial referrals for the firm are not made due to the high referral costs incurred by the referring workers. In such cases, the firm may consider introducing an ERP to externally incentivize its employees to refer their social contacts.

As discussed in the text, the empirical predictions of the model regarding the wage dynamics of referred and non-referred workers, their turnover, and the association between the characteristics of referred workers and the ERP bonus level align with the empirical evidence documented in the literature. This alignment supports the validity and credibility of the model. The study also provides testable predictions about the conditions in which voluntary referrals and ERPs are likely to happen. Furthermore, it offers insights into the benefits experienced by firms and workers from both voluntary referrals and ERPs. These findings could potentially guide future empirical research on employee and job referrals.

\singlespacing
\setlength\bibsep{0pt}
\bibliographystyle{plainnat}
\bibliography{references}



\clearpage

\onehalfspacing

% \section*{Tables} \label{sec:tab}
% \addcontentsline{toc}{section}{Tables}



% \clearpage

% \section*{Figures} \label{sec:fig}
% \addcontentsline{toc}{section}{Figures}

%\begin{figure}[hp]
%  \centering
%  \includegraphics[width=.6\textwidth]{../fig/placeholder.pdf}
%  \caption{Placeholder}
%  \label{fig:placeholder}
%\end{figure}




\clearpage

\section*{Appendix A} \label{sec:appendixa}
\addcontentsline{toc}{section}{Appendix A}
Appendix A contains derivations and proofs omitted in the text.
\subsection*{Wage derivation}
\begin{itemize}
    \item Wage of the worker $m$ from the labor market in the first period, $w_{m,1}$, defined in (\ref{eq_w_m_1}) is equal to:
    \begin{equation*}
    w_{m,1} = \mathbb{E}[y_{m}] = \mathbb{E}[\theta_m + \mu_m] = \bar{\theta}
    \end{equation*}
    
    \item Wage of the worker $m$ from the labor market in the second period, $w_{m,2}$, defined in (\ref{eq_w_m_2}) is equal to:
    \begin{equation*}
    w_{m,2} = \mathbb{E}[\theta_{m}|y_{m}] + \mathbb{E}[\mu_m] = \mathbb{E}[\theta_{m}] + \frac{Cov(\theta_m, y_m)}{\sqrt{\sigma^2(1+\sigma^2)}}\frac{\sigma(y_m - \bar{\theta})}{\sqrt{1+\sigma^2}} = \bar{\theta} + \frac{\sigma^2}{1+\sigma^2}(y_m - \bar{\theta})
    \end{equation*}
    Note that $Cov(\theta_m, y_m) = \mathbb{E}[(\theta_m - \mathbb{E}[\theta_m])(y_m - \mathbb{E}[y_m])] = \mathbb{E}[\theta_m(\theta_m + \mu_m)]-\bar{\theta}^2 = \sigma^2$ because $Cov(\theta_m,\mu_m) = 0$.
    
    \item Wage of the referred worker $j$ by the current employee $i$ with $y_i$, $w_{j,1}$, defined in (\ref{eq_w_j_1_y_i}) is equal to:
    \begin{equation}\label{eq_w_j_1}
    w_{j,1} = \mathbb{E}[\theta_{j}|y_{i}] + \mathbb{E}[\mu_j] = \mathbb{E}[\theta_j]+ \frac{Cov(\theta_j,y_i)}{1+\sigma^2}\left(y_i - \mathbb{E}[y_i]\right) = \bar{\theta}+\rho\frac{\sigma^2}{1+\sigma^2}(y_i-\bar{\theta})
    \end{equation}
    $Cov(\theta_j, y_i) = \mathbb{E}[(\theta_j - \mathbb{E}[\theta_j])(y_i - \mathbb{E}[y_i])] = \mathbb{E}[\theta_j(\theta_i + \mu_i)]-\bar{\theta}^2 = Cov(\theta_i,\theta_j)$ because $Cov(\theta_j,\mu_i) = 0$.
    Note that the labor market participants do not observe $\theta_i$, and $\mu_i$, and thus update their beliefs about the level of $\theta_j$ only based on $i$'s output $y_i$.
    
    \item Wage of the referred worker $j$ by the current employee $i$ with $y_i$ in the second period, defined in (\ref{eq_w_j_2}) is equal to: $w_{j,2} = \mathbb{E}[\theta_j|y_j,y_i]$. 
    
    Note, that $\mathbb{E}[\theta_j|y_j,y_i] = \mathbb{E}[\theta_j|y_j]$ in case when $Corr(\theta_i,\theta_j)= Corr(\mu_i,\mu_j)$. Let's denote $\Sigma_{12} = [\sigma^2 \text{    } \rho\sigma^2]$, and $\Sigma_{22}$ as:
    \begin{equation*}
        \Sigma_{22} = 
        \begin{bmatrix}
        1+\sigma^2 & \rho (1+\sigma^2)\\
        \rho (1+\sigma^2)& \sigma^2
        \end{bmatrix}
    \end{equation*}
    Then, 
    \begin{equation*}
    \mathbb{E}[\theta_j|y_j,y_i] = \mathbb{E}[\theta_j]+ \Sigma_{12}\Sigma_{22}^{-1}
    \begin{bmatrix}
        y_j-\mathbb{E}[y_j]\\
        y_i-\mathbb{E}[y_i]
    \end{bmatrix}
    \end{equation*}
    Simplifying the equation we obtain:
    \begin{equation*}
        \mathbb{E}[\theta_j|y_j,y_i] = \bar{\theta}+ \frac{\sigma^2-\rho^2\sigma^2}{(1-\rho^2)(1+\sigma^2)}(y_j - \bar{\theta})+ \frac{\rho\sigma^2-\rho\sigma^2}{(1-\rho^2)(1+\sigma^2)}(y_i - \bar{\theta})
    \end{equation*}
    Notice, that the last term on the right hand side is equal to zero. It means, that conditional expectation of $\theta_j$ depends on $y_i$ only through the value of $y_j$, i.e.:
    \begin{equation*}
        \mathbb{E}[\theta_j|y_j,y_i] = \bar{\theta}+\frac{\sigma^2}{1+\sigma^2}(y_j - \bar{\theta}) = \mathbb{E}[\theta_j|y_j]
    \end{equation*}
    
    \item The expected wage of the worker $j$ referred by the current employee $i$ in period $t_i = 2$, given the output of the current employee $y_i$, $\mathbb{E}[w_{j,2}|y_i]$:
    \begin{equation}\label{eq:w_j_2_cond_y_i}
       \mathbb{E}[w_{j,2}|y_i] =  \bar{\theta}+\frac{\sigma^2}{1+\sigma^2}(\mathbb{E}[y_j | y_i] - \bar{\theta}) = \bar{\theta}+\frac{\sigma^2}{1+\sigma^2}\rho(y_i-\bar{\theta})
    \end{equation}
    Note that the wage of worker $j$ in the second period does not take into account whether he stays in the firm or not.  This is because the current employee estimates the expected wage of her friend in the second period based on his expected output in the first period, which is correlated with her output $y_i$. % Moreover, this wage reflects the beliefs of the current employee $i$ about worker $j$'s wage in the second period, irrespective of whether she referred him or not.

    \item Let's denote the conditional output of the worker $j$ in the firm given the output of the current employee $i$ as $y'_j = \lbrace y_j | y_i \rbrace$. Then the expected wage of the worker $j$ referred by the current employee $i$ with the output $y_i$ in period $t_i = 2$, given that he stayed in the firm in the second period, denoted as $\mathbb{E}[w_{j,2}| y'_j  \geq \bar{\theta}]$, is equal to:
    \begin{equation}\label{eq:wage_exp_y'j}
        \mathbb{E}[w_{j,2}| y'_j \geq \bar{\theta}]= \bar{\theta} + \frac{\sigma^2}{1+\sigma^2}(\mathbb{E}[y'_j | y'_j \geq \bar{\theta}] - \bar{\theta})
    \end{equation}
    Note, that $\mathbb{E}[y'_j] = \mathbb{E}[y_j | y_i] = \bar{\theta} + \rho (y_i - \bar{\theta})$, and $Var(y'_j) = (1-\rho^2)(1+\sigma^2)$. Thus, after simplifying (\ref{eq:wage_exp_y'j}) the wage of the worker $j$ $\mathbb{E}[w_{j,2}|y'_j \geq \bar{\theta}]$ is equal to:
    \begin{equation}\label{eq:wage_j2_y_i}
        \mathbb{E}[w_{j,2}| y'_j \geq \bar{\theta}]= \bar{\theta} + \frac{\sigma^2}{1+\sigma^2} \left( \rho(y_i - \bar{\theta}) + \sqrt{(1+\sigma^2)(1-\rho^2)}\lambda \left( -\alpha(y_i) \right) \right),
    \end{equation}
    where $\lambda(\cdot) = \frac{\phi(\cdot)}{1-\Phi(\cdot)}$ is the inverse Mills ratio, and $\alpha(y_i) = \frac{\rho(y_i - \bar{\theta})}{\sqrt{(1+\sigma^2)(1-\rho^2)}}$
\end{itemize}

\subsection*{Profit derivation}
\begin{itemize}
    \item The firm's expected profit generated by a worker $j$ referred by a current employee in period $t_j = 1$, denoted as $\pi_{j,1}$, and defined in (\ref{eq_pi_j_1}) is equal to:
    \begin{equation*}
        \pi_{j,1} = \mathbb{E}[y_j | y_i] - w_{j,1} = \bar{\theta} + \rho (y_i - \bar{\theta}) - w_{j,1} = \frac{\rho (y_i - \bar{\theta})}{1+\sigma^2}
    \end{equation*}
    \item The firm's expected profit generated by a worker $m$ hired from the labor market, conditional on his staying in the firm for the second period, denoted as $\mathbb{E}[\pi_{m,2} | y_m \geq \bar{\theta}]$ is equal to: %and defined in (\ref{eq_pi_m_2_stayed})
    \begin{equation*}
        \mathbb{E}[\pi_{m,2} | y_m \geq \bar{\theta}] = \frac{\mathbb{E}[y_m | y_m \geq \bar{\theta}] - \bar{\theta}} {1+\sigma^2} = \frac{\sqrt{1+\sigma^2}}{1+\sigma^2}\lambda\left(\frac{\bar{\theta}-\bar{\theta}}{\sqrt{1+\sigma^2}}\right) = \frac{\lambda(0)}{\sqrt{1+\sigma^2}},
    \end{equation*}
    where $\lambda(\cdot) = \frac{\phi(\cdot)}{1-\Phi(\cdot)}$ is the inverse Mills ratio.
    \item The overall expected profit of the firm from hiring worker $m$, denoted as $
    \Pi_m$, and defined in (\ref{eq_Pi_m}) is equal to:
    \begin{equation}\label{eq_profit_m}
        \Pi_m = \pi_{m,1}+P(y_m \geq \bar{\theta})\mathbb{E}\left[\pi_{m,2}|y_m \geq \bar{\theta}\right] 
        = 0 + \left( 1- \Phi \left( 0 \right) \right) \frac{\lambda(0)}{\sqrt{1+\sigma^2}}
        = \frac{\phi(0)}{\sqrt{1+\sigma^2}}
    \end{equation}
    \item The expected profit of the firm from hiring worker $j$ referred by the current employee with the output $y_i$ in $t_j = 2$, given that the worker $j$ stays in the firm, can be denoted as $\mathbb{E}[\pi_{j,2}| y'_j \geq \bar{\theta}]$, where $y'_j = \lbrace y_j | y_i \rbrace$ is the output of the worker $j$ conditional on the realization of the output of the current employee $i$. As shown before, $y'_j \sim \mathcal{N}\left( \bar{\theta}+ \rho(y_i-\bar{\theta}), (1-\rho^2)(1+\sigma^2)\right)$. Therefore, $\mathbb{E}[\pi_{j,2}| y'_j \geq \bar{\theta}]$ is equal to:
    \begin{equation*}
        \mathbb{E}[\pi_{j,2}| y'_j \geq \bar{\theta}]
        = \frac{\mathbb{E}[y'_j| y'_j \geq \bar{\theta}]-\bar{\theta}}{1+\sigma^2}
        = \frac{1}{1+\sigma^2}\left( \rho(y_i - \bar{\theta}) + \sqrt{(1-\rho^2)(1+\sigma^2)} \lambda(-\alpha(y_i)) \right)
    \end{equation*}
    where $\alpha(y_i) = \frac{\rho\left(y_i - \bar{\theta}\right)}{\sqrt{(1-\rho^2)(1+\sigma^2)}}$. After simplification we obtain:
    \begin{equation}\label{eq:profit_j_exp_y_i}
        \mathbb{E}[\pi_{j,2}| y'_j \geq \bar{\theta}] = \frac{\rho(y_i - \bar{\theta})}{1+\sigma^2} + \sqrt{\frac{1-\rho^2}{1+\sigma^2}}\lambda(-\alpha(y_i)),
    \end{equation}
    \item The expected profit of the firm from hiring worker $j$ referred by the current employee with output $y_i$ at the moment of making hiring decision, denoted as $\Pi_j(y_i)$, is equal to:
    \begin{equation*}
        \Pi_j(y_i) = \pi_{j,1} + P\left( y'_j \geq \bar{\theta} \right) \mathbb {E}[\pi_{j,2} | y'_j \geq \bar{\theta}] 
        + (1 - P\left( y'_j \geq \bar{\theta} \right)) \pi_{m,1},
    \end{equation*}
    where $P\left( y'_j \geq \bar{\theta} \right)$ is the probability that the worker $j$'s output, conditional on the current employee's output $y_i$, is higher than $\bar{\theta}$. In other words, it is the probability that the worker $j$ will stay in the firm for $t_j = 2$. The third summand on the right-hand side denotes that the firm will hire the labor market candidate $m$ in case the worker $j$ leaves the firm before $t_j = 2$. Note that $P\left( y'_j \geq \bar{\theta} \right) = 1-\Phi(-\alpha(y_i)) = \Phi(\alpha(y_i))$. Hence, the expected profit of the firm from hiring worker $j$ is equal to:
    \begin{equation}\label{eq_pi_j_y_i}
        \Pi_j(y_i) = \frac{\rho\left(y_i-\bar{\theta}\right)}{1+\sigma^2}\left(1+\Phi\left(\alpha(y_i)\right)\right)
        + \sqrt{\frac{1-\rho^2}{1+\sigma^2}}\phi\left(\alpha(y_i)\right),
    \end{equation}
    where $\alpha(y_i) = \frac{\rho\left(y_i - \bar{\theta}\right)}{\sqrt{(1-\rho^2)(1+\sigma^2)}}$.
    \item The expected profit of the firm under ERP with bonus $b \geq 0$ from hiring worker $j$ referred by the current employee with output $y_i$ at the moment of making hiring decision, denoted as $\Pi_j(y_i, b)$, is equal to:
    \begin{equation*}
        \Pi_j(y_i, b) = \pi_{j,1} - b + P\left( y'_j \geq \bar{\theta} \right) \mathbb {E}[\pi_{j,2} | y'_j \geq \bar{\theta}] 
        + (1 - P\left( y'_j \geq \bar{\theta} \right)) \pi_{m,1},
    \end{equation*}
    where $P\left( y'_j \geq \bar{\theta} \right)$ is the probability that the worker $j$'s output, conditional on the current employee's output $y_i$, is higher than $\bar{\theta}$. Hence, the expected profit of the firm under ERP with bonus $b \geq 0$ from hiring worker $j$ is equal to:
    \begin{equation}\label{eq_pi_j_y_i_b}
        \Pi_j(y_i, b) = \frac{\rho\left(y_i-\bar{\theta}\right)}{1+\sigma^2}\left(1+\Phi\left(\alpha(y_i)\right)\right)
        + \sqrt{\frac{1-\rho^2}{1+\sigma^2}}\phi\left(\alpha(y_i)\right) - b,
    \end{equation}
    where $\alpha(y_i) = \frac{\rho\left(y_i - \bar{\theta}\right)}{\sqrt{(1-\rho^2)(1+\sigma^2)}}$.
\end{itemize}

\subsection*{Wage derivation: amended model}
\begin{itemize}
    \item Wage of the worker $m$ from the labor market in the first period in the amended model, denoted as $w'_{m,1}$, is equal to:
    \begin{equation*}
        w'_{m,1} = \mathbb{E}[y_{m}] = \mathbb{E}[\theta_m + \mu_m] = \bar{\theta}
    \end{equation*}
    
    \item Wage of the worker $m$ from the labor market in the second period in the amended model, denoted as $w'_{m,2}$, is equal to:
    \begin{equation*}
        w'_{m,2} = \mathbb{E}[y_{m}|\theta_{m}] = \mathbb{E}[\theta_{m}|\theta_m] + \mathbb{E}[\mu_{m}] = \theta_m
    \end{equation*}
    
    \item Wage of the referred worker $j$ by the current employee $i$ with $\theta_i$ and $\mu_i$, denoted as $w'_{j,1}$, is equal to:
    \begin{equation}\label{eq_ext_w_j_1}
    w'_{j,1} = \mathbb{E}[y_j|\theta_{i}] = 
    \mathbb{E}[\theta_j|\theta_{i}] + \mathbb{E}[\mu_j]
    =
    \bar{\theta}+\rho(\theta_i-\bar{\theta})
    \end{equation}
    \item Wage of the referred worker $j$ by the current employee $i$ with $\theta_i$ and $\mu_i$ in the second period in the amended model, denoted as $w'_{j,2}$ is equal to: 
    \begin{equation}
        w'_{j,2} = \mathbb{E}[\theta_j|\theta_j,\theta_i] + \mathbb{E}[\mu_j] = \theta_j
    \end{equation}
    \item The expected wage of the worker $j$ referred by the current employee $i$ in period $t_i = 2$, given the general ability level of the current employee $\theta_i$ is equal to $\mathbb{E}[w'_{j,2}|\theta_i] = \mathbb{E}[\theta_j | \theta_i] = \bar{\theta} + \rho(\theta_i-\bar{\theta})$.
\end{itemize}

\subsection*{Profit derivation: amended model}
\begin{itemize}
    \item The firm's expected profit generated by a labor market participant $m$ in period $t_m = 1$ in the amended model, denoted as $\pi'_{m,1}$ is equal to:
    \begin{equation*}
        \pi'_{m,1} = \mathbb{E}[y_m] - w'_{m,1} = \bar{\theta}-\bar{\theta} = 0
    \end{equation*}
    \item The firm's expected profit generated by a labor market participant $m$ with the general ability level $\theta_m$ and firm-specific ability level $\mu_m$ in period $t_m = 2$ in the amended model, denoted as $\pi'_{m,2}$ is equal to:
    \begin{equation*}
        \pi'_{m,2} = \mathbb{E}[y_m | \theta_m, \mu_m] - w'_{m,2} = \theta_m + \mu_m - \theta_m = \mu_m
    \end{equation*}
    \item The firm's expected profit generated by a worker $j$ referred by a current employee in period $t_j = 1$ in the amended model, denoted as $\pi'_{j,1}$, is equal to:
    \begin{equation*}
        \pi'_{j,1} = \mathbb{E}[y_j | \theta_i, \mu_i] - w'_{j,1} = \mathbb{E}[\theta_j|\theta_i] + \mathbb{E}[\mu_j|\mu_i] -w'_{j,1} = \bar{\theta} + \rho (\theta_i - \bar{\theta}) + \rho\mu_i - w'_{j,1} = \rho\mu_i
    \end{equation*}
    \item The firm's expected profit generated by a worker $j$ referred by a current employee in period $t_j = 2$ in the amended model, denoted as $\pi'_{j,2}$, is equal to:
    \begin{equation*}
        \pi'_{j,2} = \mathbb{E}[y_j | \theta_j, \mu_j] - w'_{j,2} = \mu_j
    \end{equation*}
    \item The firm's expected profit generated by a worker $m$ hired from the labor market, conditional on his staying in the firm for the second period in the amended model, denoted as $\mathbb{E}[\pi'_{m,2} | \mu_m \geq 0]$ is equal to: %and defined in (\ref{eq_pi_m_2_stayed})
    \begin{equation*}
        \mathbb{E}[\pi'_{m,2} | \mu_m \geq 0] = \mathbb{E}[\mu_m | \mu_m \geq 0] = \lambda(0),
    \end{equation*}
    where $\lambda(\cdot) = \frac{\phi(\cdot)}{1-\Phi(\cdot)}$ is the inverse Mills ratio.
    \item The overall expected profit of the firm from hiring worker $m$, denoted as $
    \Pi'_m$, is equal to:
    \begin{equation}\label{eq_ext_profit_m}
        \Pi'_m = \pi'_{m,1}+P(\mu_m \geq 0)\mathbb{E}\left[\pi'_{m,2}|\mu_m \geq 0\right] 
        = 0 + \left( 1- \Phi \left( 0 \right) \right) \lambda(0)
        = \phi(0)
    \end{equation}
    \item The expected profit of the firm from hiring worker $j$ referred by the current employee $i$ in $t_j = 2$ in the amended model, given that the worker $j$ stays in the firm, can be denoted as $\mathbb{E}[\pi'_{j,2}| \mu'_j \geq 0]$, where $\mu'_j = \lbrace \mu_j | \mu_i \rbrace$ is the firm-specific ability of the worker $j$ conditional on the realization of the firm-specific ability of the current employee $i$. $\mu'_j \sim \mathcal{N}\left( \rho\mu_i, (1-\rho^2)\right)$. Therefore, $\mathbb{E}[\pi'_{j,2}| \mu'_j \geq 0]$ is equal to:
    \begin{equation*}
        \mathbb{E}[\pi'_{j,2}| \mu'_j \geq 0]
        = \mathbb{E}[\mu'_j| \mu'_j \geq 0]
        = \rho\mu_i + \sqrt{1-\rho^2}\lambda \left(\frac{-\rho\mu_i}{\sqrt{1-\rho^2}}\right)
    \end{equation*}
    \item The expected profit of the firm from hiring worker $j$ referred by the current employee $i$ at the moment of making hiring decision in the amended model, denoted as $\Pi'_j(\mu_i)$, is equal to:
    \begin{equation*}
        \Pi'_j(\mu_i) = \pi'_{j,1} + P\left( \mu'_j \geq 0 \right) \mathbb {E}[\pi'_{j,2} | \mu'_j \geq 0] 
        + (1 - P\left( \mu'_j \geq 0 \right)) \pi'_{m,1},
    \end{equation*}
    where $P\left( \mu'_j \geq 0 \right)$ is the probability that the worker $j$'s firm-specific ability, conditional on the current employee's firm-specific ability $\mu_i$, is higher than $0$. In other words, it is the probability that the worker $j$ will stay in the firm for $t_j = 2$. The third summand on the right-hand side denotes that the firm will hire the labor market candidate $m$ in case the worker $j$ leaves the firm before $t_j = 2$. After simplification, the expected profit of the firm from hiring worker $j$ in the amended model is equal to:
    \begin{equation}\label{eq_ext_pi_j_y_i}
        \Pi'_j(\mu_i) = \rho\mu_i \left(1+\Phi\left(\frac{\rho\mu_i}{\sqrt{1-\rho^2}}\right)\right) + \sqrt{1-\rho^2}\phi\left(\frac{\rho\mu_i}{\sqrt{1-\rho^2}}\right)
    \end{equation}

    \item The expected profit of the firm under ERP with bonus $b' \geq 0$ from hiring worker $j$ referred by the current employee $i$ at the moment of making hiring decision in the amended model, denoted as $\Pi'_j(\mu_i, b')$, is equal to:
    \begin{equation*}
        \Pi'_j(\mu_i, b') = \pi'_{j,1} - b' + P\left( \mu'_j \geq 0 \right) \mathbb {E}[\pi'_{j,2} | \mu'_j \geq 0] 
        + (1 - P\left( \mu'_j \geq 0 \right)) \pi'_{m,1}
    \end{equation*}
    Thus, the expected profit of the firm under ERP with bonus $b' \geq 0$ from hiring worker $j$ in the amended model is equal to:
    \begin{equation}\label{eq_ext_pi_j_y_i_b}
        \Pi'_j(\mu_i, b') =  \rho\mu_i \left(1+\Phi\left(\frac{\rho\mu_i}{\sqrt{1-\rho^2}}\right)\right) + \sqrt{1-\rho^2}\phi\left(\frac{\rho\mu_i}{\sqrt{1-\rho^2}}\right) - b',
    \end{equation}
\end{itemize}

\begin{proof}
    \textbf{Proposition \ref{prop:eq_no_ref}}. 
    Let's start solving the game backwards and begin with period $t_m = 2$ first and then consider period $t_m = 1$. At the end of period 1, the labor market participants observe the output of the worker $m$  and update their beliefs about their expected general ability. The market wage of the worker $m$ is therefore equal to $w_{m,2} = \mathbb{E}[\theta_m | y_m] = \bar{\theta} + \rho\frac{\sigma^2}{1+\sigma^2}(y_m - \bar{\theta})$. The firm observes the market's offer and decides whether to retain the worker $m$. The firm retains the worker $m$ if its profit from retaining the worker is higher than the profit from hiring another labor market participant, i.e. when $y_m - w_{m,2} \geq \pi_{m,1}$. After simplification of the inequality we obtain the following result: the firm retains the worker $m$ if $\frac{y_m - \bar{\theta}}{1+\sigma^2} \geq 0$ which is equivalent to $y_m \geq \bar{\theta}$, given that $\sigma^2 >0$. 
    
    The profit expected by the firm at the beginning of the game from retaining worker $m$ in period $t_m = 2$ is equal to $\Pi_{m,2} = P(y_m \geq \bar{\theta})\mathbb{E}[\pi_{m,2}|y_m \geq \bar{\theta}]$, where $P(y_m \geq \bar{\theta}) = 1-\Phi(0)$ is the probability that worker $m$ stays in the second period, and $\mathbb{E}[\pi_{m,2}|y_m \geq \bar{\theta}] = \frac{\lambda(0)}{\sqrt{1+\sigma^2}}$ is the firm's expected profit generated by worker $m$ conditional on his staying with the firm for the second period.  
    
    In period $t_m = 1$ competition on the labor market implies that the firms offer each job candidate $m$ the job offer with the wage equal to his expected general ability, i.e. $w_{m,1} = \mathbb{E}[\theta_m] = \bar{\theta}$. The profit of the firm in the first period is equal to zero, because $\mathbb{E}[y_m] = \mathbb{E}[\theta_m]$. 
\end{proof}

\begin{proof}
    \textbf{Lemma \ref{lemma:y_star_existence}.}
    First, consider the difference between the firm's expected profit from employing a worker $j$ referred by the current employee $i$ with output level $y_i$, denoted as $\Pi_j(y_i)$ and the expected profit from hiring a labor market candidate $m$, denoted as $\Pi_m$. This difference, denoted as $\Delta\Pi_{j,m}(y_i)$, is equal to:
\begin{equation*}
    \Delta\Pi_{j,m}(y_i) = \Pi_j(y_i) - \Pi_m
    = \frac{\rho\left(y_i-\bar{\theta}\right)}{1+\sigma^2}\left(1+\Phi\left(\alpha(y_i)\right)\right)
    + \sqrt{\frac{1-\rho^2}{1+\sigma^2}}\phi\left(\alpha(y_i)\right) - \frac{\phi(0)}{\sqrt{1+\sigma^2}}
\end{equation*}
Note, that $\Delta\Pi_{j,m}(\bar{\theta}) = \frac{\phi(0)(\sqrt{1-\rho^2}-1)}{\sqrt{1+\sigma^2}}<0$ and $\lim_{y_i \rightarrow \infty}{\Delta\Pi_{j,m}(y_i) = \infty>0}$. Moreover, the first derivative of $\Delta\Pi_{j,m}(y_i)$ on $y_i$ is positive:
\begin{equation*}
    \frac{d\Delta\Pi_{j,m}(y_i)}{dy_i} = \frac{\rho\left( 1+ \Phi(\alpha(y_i) \right)}{1+\sigma^2}
    +
    \frac{\rho \left( y_i - \bar{\theta} \right)}{1+\sigma^2}\phi(\alpha(y_i))\frac{\partial \alpha(y_i)}{\partial y_i}
    -
    \sqrt{\frac{1-\rho^2}{1+\sigma^2}}\phi(\alpha(y_i))\alpha(y_i)\frac{\partial \alpha(y_i)}{\partial y_i}
\end{equation*}
i.e.
\begin{equation}\label{eq:deriv_delta_pi}
    \frac{d\Delta\Pi_{j,m}(y_i)}{dy_i} = \frac{\rho\left( 1+ \Phi(\alpha(y_i) \right)}{1+\sigma^2} >0 \text{   } \forall y_i \in \mathbb{R}
\end{equation}
From (\ref{eq:deriv_delta_pi}) it follows, that $\Delta\Pi_{j,m}(y_i)$ is strictly increasing in $y_i$. Therefore, there exists a unique $y^* \in (\bar{\theta},\infty)$ s.t. $\Delta\Pi_{j,m}(y^*) = 0$.
\end{proof}

\begin{proof}
   \textbf{ Lemma \ref{lemma:y_tilde_existence}.}
   Let's show that the current employee will never make a referral if her social preference parameter $\psi_{ij} = 0$. Note that $U_i(y_i, r_i = 2) = w_{i,2}-C(\bar{\theta})$ under $\psi_{ij} = 0$, while $U_i(y_i, r_i = 0) = w_{i,2}$. Hence, if the average level of the worker's general ability is positive, i.e., $\bar{\theta} >0$, then $U_i(y_i, r_i = 2) < U_i(y_i, r_i = 0)$ for all $y_i \in \mathbb{R}$ due to assumption (A8).

    The existence of $\Tilde{y}$ under $\psi_{ij}>0$ is shown in the text of the paper. Making $\Delta U_i(y_i)$ equal to zero results in finding the threshold $\Tilde{y}$:

    \begin{equation*}
        \Delta U_i(y_i) = 0 \Leftrightarrow y_i = \bar{\theta} + \frac{C(\bar{\theta})(1+\sigma^2)}{\psi_{ij}\rho\sigma^2}
    \end{equation*}

This equivalence holds under conditions that $\rho \in (0,1)$ and $\sigma^2 >0$, which are stated in  assumptions (A5) and (A9).
\end{proof}

\begin{proof}
    \textbf{Proposition \ref{prop:eq_vr}.}
    Let's start solving the game backwards and begin with period $t_j = 2$ first and then consider period $t_j = 1$. At the end of period 1, the labor market participants observe the output of the worker $j$ and update their beliefs about his expected general ability, $\theta_j$. The market wage of the worker $j$ is therefore equal to $w_{j,2} = \mathbb{E}[\theta_j | y_j] = \bar{\theta} + \rho\frac{\sigma^2}{1+\sigma^2}(y_j - \bar{\theta})$. The firm observes the market's offer and decides whether to retain the worker $j$. The firm retains the worker $j$ if its profit from retaining the worker is higher than the profit from hiring a labor market participant, i.e. when $y_j - w_{j,2} \geq \pi_{m,1}$. After simplification of the inequality we obtain the following result: the firm retains the worker $j$ if $\frac{y_j - \bar{\theta}}{1+\sigma^2} \geq 0$ which is equivalent to $y_j \geq \bar{\theta}$, given that $\sigma^2 >0$. 
    
    The profit expected by the firm at the beginning of the game from retaining worker $j$ in period $t_j = 2$ is equal to $\Pi_{j,2} = P(y'_j \geq \bar{\theta})\mathbb{E}[\pi_{j,2}|y'_j \geq \bar{\theta}]$, where $P(y'_j \geq \bar{\theta}) = 1-\Phi(-\alpha(y_i))$ is the probability that worker $j$, referred by the current employee with the output $y_i$ stays in the second period, and $\mathbb{E}[\pi_{j,2}|y'_j \geq \bar{\theta}]$ is the firm's expected profit generated by worker $j$ conditional on his staying with the firm for the second period.  
    
    In period $t_j = 1$ due to competition on the labor market, firms offer job candidate $j$, referred by the employee with the output $y_i$ the job offer with the wage equal to his expected general ability, conditional on $y_i$, i.e. $w_{j,1} = \mathbb{E}[\theta_j| y_i]$. The profit of the firm in the first period is equal to $\pi_{j,1} = \frac{\rho(y_i - \bar{\theta})}{1+\sigma^2}$. The firm hires candidate $j$ referred by the current employee with the output level $y_i$ when its overall expected profit from hiring this candidate, denoted as $\Pi_j(y_i)$ and defined in (\ref{eq_pi_j_y_i}) is higher than the overall expected profit from hiring labor market candidate $m$, denoted as $\Pi_m$ and defined in (\ref{eq_profit_m}). Lemma \ref{lemma:y_tilde_existence} completes the proof by showing that current employee $i$ refers her contact $j$ only if she is confident that the firm will hire the candidate.
\end{proof}

\begin{proof}
    \textbf{Corollary \ref{cor:wages_vr}.}
    \begin{enumerate}[label={\roman*})]
        \item The initial wage of referred candidate $j$ is equal to $w_{j,1} =\bar{\theta}+\rho\frac{\sigma^2}{1+\sigma^2}(y_i-\bar{\theta})$, while the initial wage of labor market candidate is equal to $w_{m,1} = \bar{\theta}$. According to Lemma \ref{lemma:y_tilde_existence}, $y_i \geq \bar{\theta}$, and assuming that $\rho \in (0,1)$ and $\sigma^2 >0$, as stated in assumptions (A5) and (A9), the inequality follows.
        \item The expected wage of referred worker $j$ who stayed in the firm in period $t_j = 2$, denoted as $\mathbb{E}[w_{j,2}|y'_j \geq \bar{\theta}]$, is derived in (\ref{eq:wage_j2_y_i}). The wage of worker $j$ in period $t_j = 1$ referred by the current employee $i$ with the output $y_i$, denoted as $w_{j,1}$ is derived in (\ref{eq_w_j_1}). Hence, the difference between the wages is equal to:
        \begin{equation}\label{eq:cor_1_2}
            \mathbb{E}[w_{j,2}|y'_j \geq \bar{\theta}] - w_{j,1} = 
            \sqrt{\frac{1-\rho^2}{1+\sigma^2}}\sigma^2\lambda(-\alpha(y_i)),
        \end{equation}
        where $\alpha (y_i) = \frac{\rho\left(y_i - \bar{\theta}\right)}{\sqrt{(1-\rho^2)(1+\sigma^2)}}$. The right-hand side of the equation (\ref{eq:cor_1_2}) is positive, because the inverse Mills ratio $\lambda(x)\geq 0$ for all $x \in \mathbb{R}$.
        \item The expected wage of labor market worker $m$ who stayed in the firm in period $t_m = 2$, denoted as $\mathbb{E}[w_{m,2}|y_m \geq \bar{\theta}]$, is equal to:
        \begin{equation*}
            \mathbb{E}[w_{m,2}|y_m \geq \bar{\theta}] = \bar{\theta} + \frac{\sigma^2 }{\sqrt{1+\sigma^2}}\lambda(0)
        \end{equation*}
        The wage of labor market worker $m$ in period $t_m =1$ is equal to $w_{m,1} = \bar{\theta}$. Hence, the difference between the wages is equal to:
        \begin{equation*}
            \mathbb{E}[w_{m,2}|y_m \geq \bar{\theta}] - w_{m,1}
            = \frac{\sigma^2 }{\sqrt{1+\sigma^2}}\lambda(0) >0
        \end{equation*}
        \item The difference between wage increases is given by: 
        \begin{equation*}
            \left( \mathbb{E}[w_{j,2}|y'_j \geq \bar{\theta}] - w_{j,1} \right) - 
            \left(\mathbb{E}[w_{m,2}|y_m \geq \bar{\theta}] - w_{m,1}\right) =
            \frac{\sigma^2 }{\sqrt{1+\sigma^2}}\left(\sqrt{1-\rho^2}\lambda(\alpha(y_i))- \lambda(0)\right)
        \end{equation*}
        Given that $\sqrt{1-\rho^2} < 1$, $\alpha(y_i) \geq 0$ for all $y_i \geq \bar{\theta}$, and that $\lambda(\cdot)$ is non-decreasing, we can obtain the result that $\sqrt{1-\rho^2}\lambda(-\alpha(y_i)) < \lambda(0)$.
    \end{enumerate}
\end{proof}

\begin{proof}
    \textbf{Corollary \ref{cor:retention_vr}.}
    The probability of the non-referred worker to stay in the firm for period $t_m = 2$ is equal to $P\left(y_m \geq \bar{\theta}\right) = 1-\Phi(0) = \Phi(0)$, where $\Phi(\cdot)$ is the CDF of the standard normal distribution. The probability of worker $j$ referred by the current employee $i$ with the output level $y_i$ is equal to: $P\left( y'_j \geq \bar{\theta} \right) = 1-\Phi\left( -\frac{\rho(y_i - \bar{\theta})}{\sqrt{(1-\rho^2)(1+\sigma^2)}} \right) = \Phi\left(\frac{\rho(y_i - \bar{\theta})}{\sqrt{(1-\rho^2)(1+\sigma^2)}} \right)$. According to Lemma \ref{lemma:y_tilde_existence}, $y_i \geq \bar{\theta}$, which implies $P\left( y'_j \geq \bar{\theta} \right) \geq P\left(y_m \geq \bar{\theta}\right)$.
\end{proof}

\begin{proof}
    \textbf{Corollary \ref{cor:relation_current_empl}.}
    \begin{enumerate}[label={\roman*})]
        \item The wage of worker $j$ in period $t_j = 1$ referred by the current employee $i$ with the output $y_i$, denoted as $w_{j,1}$ is derived in (\ref{eq_w_j_1}) and equal to $w_{j,1} = \bar{\theta}+\rho\frac{\sigma^2}{1+\sigma^2}(y_i-\bar{\theta})$. It is easy to observe that $w_{j,1}$ is increasing function of $y_i$, given assumptions (A5) and (A9).
        \item The wage of worker $j$ in period $t_j = 2$ referred by the current employee $i$ with the output $y_i$, denoted as $\mathbb{E}[w_{j,2}|y_i]$ is derived in (\ref{eq:w_j_2_cond_y_i}) and equal to $\mathbb{E}[w_{j,2}|y_i] = \bar{\theta}+\frac{\sigma^2}{1+\sigma^2}\rho(y_i-\bar{\theta})$. It is easy to observe that $\mathbb{E}[w_{j,2}|y_i]$ is increasing function of $y_i$, given assumptions (A5) and (A9).
        \item Probability of referred worker $j$ to stay in the firm in period $t_j = 2$ is denoted as $P\left( y'_j \geq \bar{\theta} \right)$ and is equal to $P\left( y'_j \geq \bar{\theta} \right) = \Phi\left(\frac{\rho(y_i - \bar{\theta})}{\sqrt{(1-\rho^2)(1+\sigma^2)}} \right)$. Given that $\Phi(\cdot)$ is an increasing function, and taking into considerations assumptions (A5) and (A9), we obtain, that $P\left( y'_j \geq \bar{\theta} \right)$ is increasing function of $y_i$.
    \end{enumerate}
\end{proof}

\begin{proof}
    \textbf{Lemma \ref{lemma:erp_existence}.}
    The difference between the firm's expected profit from employing a referred candidate and the expected profit from hiring a labor market candidate, evaluated at $\Tilde{y}$ is equal to:
    \begin{equation}
        \Delta\Pi_{j,m}(\Tilde{y})
    = \frac{C(\bar{\theta})}{\sigma^2\psi_{ij}}\left(1+\Phi\left(\alpha(\Tilde{y})\right)\right)
    + \sqrt{\frac{1-\rho^2}{1+\sigma^2}}\phi\left(\alpha(\Tilde{y})\right) - \frac{\phi(0)}{\sqrt{1+\sigma^2}},
    \end{equation}
    where $\alpha(\Tilde{y}) = \sqrt{\frac{1+\sigma^2}{1-\rho^2}}\frac{C(\bar{\theta})}{\sigma^2 \psi_{ij}}$. Taking derivative of $\Delta\Pi_{j,m}(\Tilde{y})$ with respect to $\bar{\theta}$ provides us with the following result:
    \begin{equation*}
        \begin{aligned}
            \frac{d\Delta\Pi_{j,m}(\Tilde{y})}{d\bar{\theta}} & 
            = \frac{C'(\bar{\theta})}{\sigma^2 \psi_{ij}} \left(1+\Phi\left(\alpha(\Tilde{y})\right)\right)
            + \frac{C(\bar{\theta})}{\sigma^2 \psi_{ij}}\phi\left(\alpha(\Tilde{y})\right)\frac{\partial \alpha(\Tilde{y})}{\partial \bar{\theta}}
            - \sqrt{\frac{1-\rho^2}{1+\sigma^2}}\phi\left(\alpha(\Tilde{y})\right)\alpha(\Tilde{y})\frac{\partial \alpha(\Tilde{y})}{\partial \bar{\theta}}\\
            & = \frac{C'(\bar{\theta})}{\sigma^2 \psi_{ij}} \left(1+\Phi\left(\alpha(\Tilde{y})\right)\right),
        \end{aligned}
    \end{equation*}
    which is positive for $\psi_{ij} >0$ given the assumptions (A5), (A8) and (A9). Taking derivative of $\Delta\Pi_{j,m}(\Tilde{y})$ with respect to $\rho$ provides us with the following result:
    \begin{equation*}
        \begin{aligned}
            \frac{d\Delta\Pi_{j,m}(\Tilde{y})}{d\rho} & 
            = \frac{C(\bar{\theta})}{\sigma^2 \psi_{ij}} \phi\left(\alpha(\Tilde{y})\right) \frac{\partial \alpha(\Tilde{y})}{\partial \rho}
            -\frac{\rho \phi\left(\alpha(\Tilde{y})\right)}{\sqrt{(1+\sigma^2)(1-\rho^2)}}
            - \sqrt{\frac{1-\rho^2}{1+\sigma^2}} \phi\left(\alpha(\Tilde{y})\right) \alpha(\Tilde{y}) \frac{\partial \alpha(\Tilde{y})}{\partial \rho}\\
            & = -\frac{\rho \phi\left(\alpha(\Tilde{y})\right)}{\sqrt{(1+\sigma^2)(1-\rho^2)}},
        \end{aligned}
    \end{equation*}
    which is negative given the assumptions (A5) and (A9). Taking derivative of $\Delta\Pi_{j,m}(\Tilde{y})$ with respect to $\psi_{ij}$ provides us with the following result:
    \begin{equation*}
        \begin{aligned}
            \frac{d\Delta\Pi_{j,m}(\Tilde{y})}{d\psi_{ij}} & 
            = -\frac{C(\bar{\theta})}{\sigma^2 \psi^2_{ij}} \left(1+\Phi\left(\alpha(\Tilde{y})\right)\right)
            + \frac{C(\bar{\theta})}{\sigma^2 \psi_{ij}}\phi\left(\alpha(\Tilde{y})\right)\frac{\partial \alpha(\Tilde{y})}{\partial \psi_{ij}}
            - \sqrt{\frac{1-\rho^2}{1+\sigma^2}}\phi\left(\alpha(\Tilde{y})\right)\alpha(\Tilde{y})\frac{\partial \alpha(\Tilde{y})}{\partial \psi_{ij}}\\
            & = -\frac{C(\bar{\theta})}{\sigma^2 \psi^2_{ij}} \left(1+\Phi\left(\alpha(\Tilde{y})\right)\right),
        \end{aligned}
    \end{equation*}
    which is negative for $\psi_{ij} >0$ given the assumptions (A5), (A8) and (A9). 
    % Taking derivative of $\Delta\Pi_{j,m}(\Tilde{y})$ with respect to $\sigma^2$ provides us with the following result:
    % \begin{equation*}
    %     \begin{aligned}
    %         \frac{d\Delta\Pi_{j,m}(\Tilde{y})}{d\psi_{ij}} & 
    %         = -\frac{C(\bar{\theta})}{\sigma^4 \psi^2_{ij}} \left(1+\Phi\left(\alpha(\Tilde{y})\right)\right)
    %         - \frac{\phi\left(\alpha(\Tilde{y})\right)}{2(1+\sigma^2)}\sqrt{\frac{1-\rho^2}{1+\sigma^2}} 
    %         + \frac{\phi(0)}{2(1+\sigma^2)\sqrt{1+\sigma^2}},
    %     \end{aligned}
    % \end{equation*}
\end{proof}

\begin{proof}
    \textbf{Proposition \ref{prop:eq_erp}.}
    Let's start solving the game backwards and begin with period $t_j = 2$ first and then consider period $t_j = 1$. At the end of period 1, the labor market participants observe the output of the worker $j$ and update their beliefs about his expected general ability, $\theta_j$. The market wage of the worker $j$ is therefore equal to $w_{j,2} = \mathbb{E}[\theta_j | y_j] = \bar{\theta} + \rho\frac{\sigma^2}{1+\sigma^2}(y_j - \bar{\theta})$. The firm observes the market's offer and decides whether to retain the worker $j$. The firm retains the worker $j$ if its profit from retaining the worker is higher than the profit from hiring a labor market participant, i.e. when $y_j - w_{j,2} \geq \pi_{m,1}$. After simplification of the inequality we obtain the following result: the firm retains the worker $j$ if $\frac{y_j - \bar{\theta}}{1+\sigma^2} \geq 0$ which is equivalent to $y_j \geq \bar{\theta}$, given that $\sigma^2 >0$. 
    
    The profit expected by the firm at the beginning of the game from retaining worker $j$ in period $t_j = 2$ is equal to $\Pi_{j,2} = P(y'_j \geq \bar{\theta})\mathbb{E}[\pi_{j,2}|y'_j \geq \bar{\theta}]$, where $P(y'_j \geq \bar{\theta}) = 1-\Phi(-\alpha(y_i))$ is the probability that worker $j$, referred by the current employee with the output $y_i$ stays in the second period, and $\mathbb{E}[\pi_{j,2}|y'_j \geq \bar{\theta}]$ is the firm's expected profit generated by worker $j$ conditional on his staying with the firm for the second period.  
    
    In period $t_j = 1$ competition on the labor market implies that the firms offer job candidate $j$, who was referred by the employee with the output $y_i$ the job offer with the wage equal to his expected general ability, i.e. $w_{j,1} = \mathbb{E}[\theta_j| y_i]$. The profit of the firm in the first period is equal to $\pi_{j,1}(b^*) = \frac{\rho(y_i - \bar{\theta})}{1+\sigma^2}-b^*$. The firm hires candidate $j$ referred by the current employee with the output level $y_i$ when its overall expected profit from hiring this candidate, denoted as $\Pi_j(y_i, b^*)$ and defined in (\ref{eq_pi_j_y_i_b}) is higher than the overall expected profit from hiring labor market candidate $m$, denoted as $\Pi_m$ and defined in (\ref{eq_profit_m}).

    The current employee $i$ refers her contact $j$ if her output level $y_i$ is higher or equal to her threshold output level $\Tilde{y}(b^*)$, where $\Tilde{y}(b^*)$ is defined in (\ref{eq_tilde_y_b}). Note, that under optimal bonus $b^*$ the firm's threshold cannot be larger than the employee's threshold, i.e. under ERP the following inequality always holds: $\Tilde{y}(b^*) \geq y^*(b^*)$. This happens, because the current employee observes the threshold of the firm and will never refer her friend if her output level $y_i < y^*(b)$. Thus, by decreasing the bonus level until $y^*(b) = \Tilde{y}(b)$ the firm can increase the profit from hiring referred candidate, $\Pi_j(y_i, b)$, while holding probability of referral, $P(y \geq \max \lbrace \Tilde{y}(b), y^*(b)\rbrace)$ constant, and thus increasing its overall profit $\Pi(b)$, defined in (\ref{eq:profit_overal_erp}).
\end{proof}

\begin{proof}
    \textbf{Corollary \ref{cor:erp_emp_evidence}.}
    \begin{enumerate}[label={\roman*})]
        \item The probability of referral under ERP with bonus $b\geq 0$ is denoted as $P\bigl(y_i \geq \Tilde{y}(b)\bigr)$ and is equal to:
        \begin{equation*}
            P\bigl(y_i \geq \Tilde{y}(b)\bigr) = 1-\Phi\left( \frac{\Tilde{y}(b) - \bar{\theta}}{\sqrt{1+\sigma^2}} \right)
        \end{equation*}
        Hence, the probability of referral increases with $b$, because $\Tilde{y}(b)$ is a decreasing function of $b$, and $\Phi(x)$ is an increasing function of $x$ for any $x$.
        \item The average expected output of the referred worker $j$ is denoted as $\mathbb{E}[y_j | y_i \geq \Tilde{y}(b)]$ and is equal to:
        \begin{equation*}
            \mathbb{E}[y_j | y_i \geq \Tilde{y}(b)] = \bar{\theta} + \rho \sqrt{1+\sigma^2}\lambda\left( \frac{\Tilde{y}(b) - \bar\theta}{\sqrt{1+\sigma^2}} \right) 
        \end{equation*}
        Hence, the probability of referral decreases with $b$, because $\Tilde{y}(b)$ is a decreasing function of $b$, and $\lambda(x)$ is an increasing function of $x$ for any $x$.
        \item The initial average wage of the referred workers is denoted as $\mathbb{E}[w_{j,1} | y_i \geq \Tilde{y}(b)]$ and is equal to:
        \begin{equation*}
            \mathbb{E}[w_{j,1} | y_i \geq \Tilde{y}(b)] = \theta + \rho\frac{\sigma^2}{\sqrt{1+\sigma^2}}\lambda\left( \frac{\Tilde{y}(b) - \bar\theta}{\sqrt{1+\sigma^2}} \right) 
        \end{equation*}
        Hence, the initial average wage of the referred workers decreases with $b$, because $\Tilde{y}(b)$ is a decreasing function of $b$, and $\lambda(x)$ is an increasing function of $x$ for any $x$.
    \end{enumerate}
\end{proof}

\begin{proof}
    \textbf{Proposition \ref{prop:ext_eq_nr}.}
    Let's start by solving the game backwards, beginning with period $t_m = 2$ and then considering period $t_m = 1$. At the end of period 1, labor market participants observe the output, general ability level $\theta_m$, and firm-specific ability $\mu_m$ of worker $m$ and update their beliefs about expected general ability. The market wage for worker $m$ is then given by $w'_{m,2} = \mathbb{E}[\theta_m | \theta_m] = \theta_m$.

    The firm observes the market's offer and decides whether to retain worker $m$. The firm retains worker $m$ if its profit from retention is higher than the profit from hiring another labor market participant, i.e., when $y_m - w'_{m,2} \geq \pi'_{m,1}$. Simplifying the inequality, we obtain the following result: the firm retains worker $m$ if $\theta_m + \mu_m - \theta_m \geq 0$, which is equivalent to $\mu_m \geq 0$.

    The expected profit for the firm from retaining worker $m$ in period $t_m = 2$ is given by $\Pi'_{m,2} = P(\mu_m \geq 0)\mathbb{E}[\pi'_{m,2}|\mu_m \geq 0]$, where $P(\mu_m \geq 0) = 1- \Phi(0)$ represents the probability that worker $m$ remains in the second period, and $\mathbb{E}[\pi'_{m,2}|\mu_m \geq 0] = \lambda(0)$ is the firm's expected profit generated by worker $m$ conditional on their continued employment for the second period.

    In period $t_m = 1$, competition in the labor market implies that firms offer each job candidate $m$ a job with a wage equal to their expected general ability, i.e., $w'_{m,1} = \mathbb{E}[\theta_m] = \bar{\theta}$. The profit of the firm in the first period is zero, as $\mathbb{E}[y_m] = \mathbb{E}[\theta_m]$.
\end{proof}

\begin{proof}
    \textbf{Lemma \ref{lemma:mu_star_existence}.}
    First, consider the difference between the firm's expected profit from employing a worker $j$ referred by the current employee $i$, denoted as $\Pi'_j(\mu_i)$ and the expected profit from hiring a labor market candidate $m$, denoted as $\Pi'_m$. This difference, denoted as $\Delta\Pi'_{j,m}(\mu_i)$, is equal to:
\begin{equation*}
    \Delta\Pi'_{j,m}(\mu_i) = \Pi'_j(\mu_i) - \Pi'_m
    = \rho\mu_i \left(1+\Phi\left(\frac{\rho\mu_i}{\sqrt{1-\rho^2}}\right)\right) + \sqrt{1-\rho^2}\phi\left(\frac{\rho\mu_i}{\sqrt{1-\rho^2}}\right) - \phi(0)
\end{equation*}
Note, that $\Delta\Pi'_{j,m}(0) = \phi(0)(\sqrt{1-\rho^2}-1)<0$ and $\lim_{\mu_i \rightarrow \infty}{\Delta\Pi'_{j,m}(\mu_i) = \infty>0}$. Moreover, the first derivative of $\Delta\Pi'_{j,m}(\mu_i)$ on $\mu_i$ is positive:

\begin{equation}\label{eq:ext_deriv_delta_pi}
    \frac{d\Delta\Pi'_{j,m}(\mu_i)}{d\mu_i} = \rho\left( 1+ \Phi\left(\frac{\rho\mu_i}{\sqrt{1-\rho^2}}\right)\right) >0 \text{   } \forall \mu_i \in \mathbb{R}
\end{equation}
From (\ref{eq:ext_deriv_delta_pi}) it follows, that $\Delta\Pi'_{j,m}(\mu_i)$ is strictly increasing in $\mu_i$. Therefore, there exists a unique $\mu^* \in (0,\infty)$ s.t. $\Delta\Pi'_{j,m}(\mu^*) = 0$.
\end{proof}

\begin{proof}
    \textbf{Lemma \ref{lemma:theta_star_existence}.}
    Let's show that the current employee will never make a referral if her social preference parameter $\psi_{ij} = 0$. Note that $U_i(\theta_i, r_i = 2) = w'_{i,2}-C(\bar{\theta})$ under $\psi_{ij} = 0$, while $U_i(\theta_i, r_i = 0) = w'_{i,2}$. Hence, if the average level of the worker's general ability is positive, i.e., $\bar{\theta} >0$, then $U_i(\theta_i, r_i = 2) < U_i(\theta_i, r_i = 0)$ for all $\theta_i \in \mathbb{R}$ due to assumption (A8).

    The existence of $\theta^*$ under $\psi_{ij}>0$ can be shown similar to the existence of $\Tilde(y)$ in the proof of Lemma \ref{lemma:y_tilde_existence}. Making $\Delta U_i(\theta_i)$ equal to zero results in finding the threshold $\theta^*$:

    \begin{equation*}
        \Delta U_i(\theta_i) = 0 \Leftrightarrow \theta_i = \bar{\theta} + \frac{C(\bar{\theta})}{\psi_{ij}\rho}
    \end{equation*}

This equivalence holds under conditions that $\rho \in (0,1)$, which is stated in Assumption A9.
\end{proof}

\begin{proof}
    \textbf{Proposition \ref{prop:ext_eq_vr}.}
    Let's start solving the game backwards and begin with period $t_j = 2$ first and then consider period $t_j = 1$. At the end of period 1, the labor market participants observe the general ability and the firm-specific ability of the worker $j$. The market wage of the worker $j$ is therefore equal to $w'_{j,2} = \theta_j$. The firm observes the market's offer and decides whether to retain the worker $j$. The firm retains the worker $j$ if its profit from retaining the worker is higher than the profit from hiring a labor market participant, i.e. when $y_j - w'_{j,2} \geq \pi'_{m,1}$. Thus, the firm retains the worker $j$ if $\mu_j \geq 0$. 
    
    The profit expected by the firm at the beginning of the game from retaining worker $j$ in period $t_j = 2$ is equal to $\Pi'_{j,2} = P(\mu'_j \geq \bar{\theta})\mathbb{E}[\pi'_{j,2}|\mu'_j \geq 0]$, where $P(\mu'_j \geq 0)$ is the probability that worker $j$, referred by the current employee $i$ stays in the second period, and $\mathbb{E}[\pi'_{j,2}|\mu'_j \geq 0]$ is the firm's expected profit generated by worker $j$ conditional on his staying with the firm for the second period.  
    
    In period $t_j = 1$ due to competition on the labor market, firms offer job candidate $j$, referred by the employee $i$ the job offer with the wage equal to his expected general ability, conditional on $\theta_i$, i.e. $w'_{j,1} = \mathbb{E}[\theta_j| \theta_i]$. The profit of the firm in the first period is equal to $\pi'_{j,1} = \rho\mu_i$. The firm hires candidate $j$ referred by the current employee $i$ when its overall expected profit from hiring this candidate, denoted as $\Pi'_j(\mu_i)$, is higher than the overall expected profit from hiring labor market candidate $m$, denoted as $\Pi'_m$. Lemmas \ref{lemma:mu_star_existence} and \ref{lemma:theta_star_existence} complete the proof by showing that current employee $i$ refers her contact $j$ only if she is confident that the firm will hire the candidate.
\end{proof}

\begin{proof}
    \textbf{Corollary \ref{cor:profit_vr_amended}.}
    Note, that $1 - \Phi\left(\frac{C(\bar{\theta})}{\rho\phi_{ij}\sigma}\right)$ is decreasing in $\bar{\theta}$, increasing in $\rho$, $\phi_{ij}$, and $\sigma$. Note also, that the right-hand side of equation (\ref{eq:profit_vr_amended}) does not depend on $\bar{\theta}$, $\phi_{ij}$, and $\sigma$. Let's show, that the right-hand side of equation (\ref{eq:profit_vr_amended}) is always positive. For that rewrite it in the following way:
    \begin{equation*}
        \int^\infty_{\mu^*}\Pi'_j(t)\phi(t)dt 
        -
        \Bigl( 1 - \Phi (\mu^*)\Bigr) \phi(0) 
        = 
        \int^\infty_{\mu^*}\left(\Pi'_j(t)-\phi(0)\right)\phi(t)dt 
    \end{equation*}
    Note also, that $\Pi'_{j}(\mu_i)$ is higher than $\phi(0)$ for all levels of $\mu_i \geq \mu^*$ according to Lemma \ref{lemma:mu_star_existence}, from which follows that the right-hand side of equation (\ref{eq:profit_vr_amended}) is always positive.
\end{proof}

\begin{proof}
    \textbf{Proposition \ref{prop:ext_eq_erp}.}
    Let's start solving the game backwards and begin with period $t_j = 2$ first and then consider period $t_j = 1$. At the end of period 1, the labor market participants observe the general and firm-specific ability of the worker $j$. The market wage of the worker $j$ is therefore equal to $w'_{j,2} = \theta_j$. The firm observes the market's offer and decides whether to retain the worker $j$. The firm retains the worker $j$ if its profit from retaining the worker is higher than the profit from hiring a labor market participant, i.e. when  $\mu_j \geq 0$. 
    
    The profit expected by the firm at the beginning of the game from retaining worker $j$ in period $t_j = 2$ is equal to $\Pi'_{j,2} = P(\mu'_j \geq 0)\mathbb{E}[\pi'_{j,2}|\mu'_j \geq 0]$, where $P(\mu'_j \geq 0)$ is the probability that worker $j$, referred by the current employee $i$ stays in the second period, and $\mathbb{E}[\pi'_{j,2}|\mu'_j \geq 0]$ is the firm's expected profit generated by worker $j$ conditional on his staying with the firm for the second period.  
    
    In period $t_j = 1$ competition on the labor market implies that the firms offer job candidate $j$, who was referred by the employee $i$ the job offer with the wage equal to his expected general ability, i.e. $w'_{j,1} = \mathbb{E}[\theta_j| \theta_i]$. The profit of the firm in the first period is equal to $\pi'_{j,1}(b^*) = \frac{\rho(y_i - \bar{\theta})}{1+\sigma^2}-b^*$. The firm hires candidate $j$ referred by the current employee with the output level $y_i$ when its overall expected profit from hiring this candidate, denoted as $\Pi_j(y_i, b^*)$ and defined in (\ref{eq_pi_j_y_i_b}) is higher than the overall expected profit from hiring labor market candidate $m$, denoted as $\Pi_m$ and defined in (\ref{eq_profit_m}).

    The current employee $i$ refers her contact $j$ if her general ability level $\theta_i$ is higher or equal to her threshold ability level $\theta^*(\Tilde{b})$, where $y^*(\Tilde{b}) = \bar{\theta} + \frac{C(\bar{\theta}) - \Tilde{b}}{\rho\\hi_{ij}}$, and her firm-specific ability level $\mu_i$ is higher or equal to her threshold firm-specific ability level $\mu_j \geq \mu^*(\Tilde{b})$. If $\theta_j < \theta^*(\Tilde{b})$, the employee's utility under referral is lower than that under no referral. If $\mu_i < \mu^*(\Tilde{b})$, the firm's profit from referred candidate is lower than that from labor market candidate.

    The firm defines the optimal bonus $\Tilde{b}$ as follows: $\Tilde{b}=\max\{0,\arg\max_{b}\Pi'(b)\}$. It ensures the maximization of the firm's profit level under positive bonus $\Tilde(b)$ in case there exists $b \geq 0$ such that $\Pi'(b) \geq \Pi'(VR)$, and forces the firm not to lunch the ERP otherwise.
\end{proof}

\pagebreak




\section*{Appendix B. Special case} \label{sec:appendixb}
\addcontentsline{toc}{section}{Appendix B}
This section presents the analysis of a special case of the model where the referral cost function, $C(\cdot)$, takes the following form: $C(\bar{\theta}) = 0.01 \bar{\theta}^2 + 0.01 \bar{\theta} + 0.01$. The following figures show the computed values for the firm's profit under no referrals ($\Pi (NR)$), under voluntary referrals ($\Pi(VR)$), and under an ERP with optimal bonus $b^*$ ($\Pi(ERP)$). They also illustrate the dynamics of referral thresholds $\Tilde{y}$ and $y^*$ under voluntary referrals, as well as their counterparts under the case of ERP ($\Tilde{y}(b^*)$ and $y^*(b^*)$, respectively). In addition, the figures demonstrate the dynamics of the optimal bonus $b^*$ in comparison to the referral costs $C(\bar{\theta})$, as well as the changes in the probabilities of making referrals under voluntary referrals and under ERP.

\begin{figure}[ht]
    \caption{Changes in model variables as average general ability $\bar{\theta}$ varies}
    \includegraphics[width=12cm]{images/imperf_means_var.png}
    \centering
    \label{fig:mean_var}
\end{figure}

Figure \ref{fig:mean_var} illustrates the changes in the aforementioned variables of the model as the average general ability level of the workers, denoted as $\bar{\theta}$, varies from 0 to 8. Other model parameters are held constant at the following fixed values: the standard deviation of the general ability level is $\sigma = 1$, the correlation between workers is $\rho = 0.5$, and the social preference parameter is $\psi_{ij} = 0.5$.

As shown in Figure \ref{fig:mean_var}, under the given parameters, introducing an employee referral program makes sense for the firm when the average level of the workers' general ability is higher than 2. While the overall profit of the firm decreases with the average level of worker's general ability, the introduction of an ERP helps to mitigate this decrease caused by the decreasing probability of referral, and it extracts additional profits compared to the case of voluntary referrals. It is worth noting that the optimal bonus the firm pays to the referring worker is always lower than her referral costs.

\begin{figure}[ht]
    \caption{Changes in model variables as standard deviation of general ability $\sigma$ varies}
    \includegraphics[width=12cm]{images/imperf_st_dev_var.png}
    \centering
    \label{fig:st_dev_var}
\end{figure}

Figure \ref{fig:st_dev_var} illustrates the changes in the  variables of the model as the standard deviation of the general ability of the workers, denoted as $\sigma$, varies from 0.2 to 2. Other model parameters are held constant at the following fixed values: the mean of the general ability level is $\bar{\theta} = 3$, the correlation between workers is $\rho = 0.5$, and the social preference parameter is $\psi_{ij} = 0.5$.

As shown in Figure \ref{fig:st_dev_var}, under the given parameters, introducing an employee referral program makes sense for the firm when the standard deviation of the general  ability of workers is lower than 1.25. The firm extracts larger benefits from voluntary referrals when there is a moderate variation in the general ability of the workers. However, the introduction of an employee referral program helps the firm increase its profit even when the variance of the general ability is low.

\begin{figure}[ht]
    \caption{Changes in model variables as correlation between workers $\rho$ varies}
    \includegraphics[width=12cm]{images/imperf_rho_var.png}
    \centering
    \label{fig:rho_var}
\end{figure}

Figure \ref{fig:rho_var} illustrates the changes in the  variables of the model as the correlation between workers, denoted as $\rho$, varies from 0.05 to 1. Other model parameters are held constant at the following fixed values: the mean of the general ability level is $\bar{\theta} = 3$, the standard deviation of the general ability level is $\sigma = 1$, and the social preference parameter is $\psi_{ij} = 0.5$. As shown in Figure \ref{fig:rho_var}, under the given parameters, introducing an employee referral program makes sense for the firm when the correlation between workers is lower than 0.85.

\begin{figure}[ht]
    \caption{Changes in model variables as social preference parameter $\psi_{ij}$ varies}
    \includegraphics[width=12cm]{images/imperf_psi_var.png}
    \centering
    \label{fig:psi_var}
\end{figure}

Figure \ref{fig:psi_var} illustrates the changes in the  variables of the model as the social preference parameter, denoted as $\psi_{ij}$, varies from 0.05 to 1. Other model parameters are held constant at the following fixed values: the mean of the general ability level is $\bar{\theta} = 3$, the standard deviation of the general ability level is $\sigma = 1$, and the correlation between workers is $\rho = 0.5$. As shown in Figure \ref{fig:psi_var}, under the given parameters, introducing an employee referral program makes sense for the firm when the social preference parameter is lower than 0.72.

Figures \ref{fig:rho_var} and \ref{fig:psi_var} indicate that an ERP is most effective for weak ties between workers. Specifically, it helps maintain a high probability of referral when the intrinsic motivation of current employees is not sufficient for them to refer their contacts. Figure \ref{fig:psi_var} also shows that under the given parameters, the ERP works best for a low level of the social preference parameter $\psi_{ij}$. Meanwhile, Figure \ref{fig:rho_var} suggests that the additional benefits from referral when the correlation of workers is low are marginal. This result is intuitive because higher correlation between workers' abilities directly affects the firm's profit, whereas changes in the social preference parameter impact the firm's profit only through the threshold of current employees to refer their contacts and are thus less salient for the firm's profit.
\pagebreak



\section*{Appendix C. Special case: amended model} \label{sec:appendixc}
\addcontentsline{toc}{section}{Appendix B}
This section presents the analysis of a special case of the amended model where the referral cost function, $C(\cdot)$, takes the following form: $C(\bar{\theta}) = 0.01 \bar{\theta}^2 + 0.01 \bar{\theta} + 0.01$. The following figures show the computed values for the firm's profit under no referrals ($\Pi' (NR)$), under voluntary referrals ($\Pi'(VR)$), and under an ERP with optimal bonus $\Tilde{b}$ ($\Pi'(ERP)$). They also illustrate the dynamics of referral thresholds $\theta^*$ and $\mu^*$ under voluntary referrals, as well as their counterparts under the case of ERP ($\theta^*(b^*)$ and $\mu^*(b^*)$, respectively). In addition, the figures demonstrate the dynamics of the optimal bonus $\Tilde{b}$ in comparison to the referral costs $C(\bar{\theta})$, as well as the changes in the probabilities of making referrals under voluntary referrals and under ERP.

\begin{figure}[ht]
    \caption{Changes in amended model variables as average general ability $\bar{\theta}$ varies}
    \includegraphics[width=12cm]{images/perf_means_var.png}
    \centering
    \label{fig:ext_mean_var}
\end{figure}

Figure \ref{fig:ext_mean_var} illustrates the changes in the aforementioned variables of the model as the average general ability level of the workers, denoted as $\bar{\theta}$, varies from 0 to 8. Other model parameters are held constant at the following fixed values: the standard deviation of the general ability level is $\sigma = 2$, the correlation between workers is $\rho = 0.5$, and the social preference parameter is $\psi_{ij} = 0.5$.

As shown in Figure \ref{fig:ext_mean_var}, under the given parameters, introducing an employee referral program makes sense for any values of the average level of the workers' general ability in the interval $[0, 8]$. While the overall profit of the firm decreases with the average level of worker's general ability, the introduction of an ERP helps to mitigate this decrease caused by the decreasing probability of referral, and it extracts additional profits compared to the case of voluntary referrals. The optimal bonus the firm pays to the referring worker can be both higher and lower than the referral costs of referring employee.

\begin{figure}[ht]
    \caption{Changes in amended model variables as standard deviation of general ability $\sigma$ varies}
    \includegraphics[width=12cm]{images/perf_st_dev_var.png}
    \centering
    \label{fig:ext_st_dev_var}
\end{figure}

Figure \ref{fig:ext_st_dev_var} illustrates the changes in the  variables of the amended model as the standard deviation of the general ability of the workers, denoted as $\sigma$, varies from 0.2 to 5. Other model parameters are held constant at the following fixed values: the mean of the general ability level is $\bar{\theta} = 3$, the correlation between workers is $\rho = 0.5$, and the social preference parameter is $\psi_{ij} = 0.5$.

As shown in Figure \ref{fig:ext_st_dev_var}, under the given parameters, introducing an employee referral program makes sense for the firm when the standard deviation of the general  ability of workers is lower than 2.4. The firm extracts larger benefits from voluntary referrals when there is a high variation in the general ability of the workers. However, the introduction of an employee referral program helps the firm increase its profit when the variance of the general ability is low.

\begin{figure}[ht]
    \caption{Changes in model variables as correlation between workers $\rho$ varies}
    \includegraphics[width=12cm]{images/perf_rho_var.png}
    \centering
    \label{fig:ext_rho_var}
\end{figure}

Figure \ref{fig:ext_rho_var} illustrates the changes in the  variables of the amended model as the correlation between workers, denoted as $\rho$, varies from 0.05 to 1. Other model parameters are held constant at the following fixed values: the mean of the general ability level is $\bar{\theta} = 3$, the standard deviation of the general ability level is $\sigma = 2$, and the social preference parameter is $\psi_{ij} = 0.5$. As shown in Figure \ref{fig:ext_rho_var}, under the given parameters, introducing an employee referral program makes sense for the firm for every value of correlation $\rho \in (0, 1)$.

\begin{figure}[ht]
    \caption{Changes in model variables as social preference parameter $\psi_{ij}$ varies}
    \includegraphics[width=12cm]{images/perf_psi_var.png}
    \centering
    \label{fig:ext_psi_var}
\end{figure}

Figure \ref{fig:ext_psi_var} illustrates the changes in the  variables of the model as the social preference parameter, denoted as $\psi_{ij}$, varies from 0.05 to 1. Other model parameters are held constant at the following fixed values: the mean of the general ability level is $\bar{\theta} = 3$, the standard deviation of the general ability level is $\sigma = 2$, and the correlation between workers is $\rho = 0.5$. As shown in Figure \ref{fig:ext_psi_var}, under the given parameters, introducing an employee referral program makes sense for the firm when the social preference parameter is lower than 0.6.

Figures \ref{fig:ext_rho_var} and \ref{fig:ext_psi_var} indicate that an ERP is most effective for weak ties between workers. Specifically, it helps maintain a high probability of referral when the intrinsic motivation of current employees is not sufficient for them to refer their contacts. Figure \ref{fig:ext_psi_var} also shows that under the given parameters, the ERP works best for a low level of the social preference parameter $\psi_{ij}$. Meanwhile, Figure \ref{fig:ext_rho_var} suggests that the additional benefits from referral when the correlation of workers is low are marginal. 

It is worth noting that the firm benefits the most from the introduction of the ERP when the ties between workers are weak (i.e., $\psi_{ij}$ is close to zero) and the hiring mechanisms are efficient (i.e., $\sigma$ is low). 

\pagebreak

\end{document}